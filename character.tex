\chapter{The Characters}

\section{Characters}

Every player character is familiar with all of the various feats of skulduggery represented by the \textbf{actions} of the game. They’re all able to \kw{Skirmish} in a knife-fight, \kw{Prowl} in the shadows, \kw{Attune} to strange energy, \kw{Consort} with contacts for information, and so on.

Of course, one will also have your specializations and skills, the qualities that make one's character uniquely effective. One might want the ability to compel obedience from demons and channel arcane energy through their body, or maybe one wants to manipulate the network of the underworld to ones advantage and see danger before it strikes, or maybe one just wants to be the deadliest fighter with a blade. In this section, one learns how to create their own unique scoundrel and choose the abilities that suit one's preferred style of play.

\section{Character Creation}

\subsection{Create Playbooks}

A playbook is what we call the sheet with all the specific rules to play a certain character type in a Blades-powered game. For example, you might create a playbook called a \kw{Soldier}, with special abilities related to battle, or a playbook called a \kw{Medic}, with special abilities related to field medicine.

When you choose a playbook, you’re choosing a set of \textbf{special abilities} (which give your character ways to break the rules in various ways) and a set of \kw{xp} \textbf{triggers} (which determine how you earn experience for character advancement). But every playbook represents a scoundrel at heart. The Cutter has special abilities related to combat, but that doesn’t mean they’re ``the fighter'' of the game. Any character type can fight well. Think of your playbook as an area of focus and preference, but not a unique skill set.

This is why we call them ``playbooks'' rather than ``character classes'' or ``archetypes.'' You’re selecting the set of initial action ratings and special abilities that your character has access to---but you’re not defining their immutable essence or true nature. Your character will grow and change over time; who they become is part of the fun of playing the game.

Once you’ve chosen your playbook, follow the steps below to complete your character.

\subsection{Choose a Heritage}

While a clan is technically all family, they have branches and holdings all over. Your character’s \textbf{heritage} describes where their family line is from. When you choose a heritage, write a detail about your family life on the line above.

\subsection{Choose a Background}

One's character's \textbf{background} describes what they did growing up, or before becoming active in Clan business. Choose a background and then write a detail about it that's specific to one's character.

\subsection{Assign Four Action Dots}

Your playbook begins with three action dots already placed. You get to add four more dots (so you’ll have seven total). At the start of the game, no action rating may have more than two dots (unless a special ability tells you otherwise). Assign your four dots like this:

\begin{itemize}
\item Put one dot in any action that you feel reflects your character’s heritage.
\item Put one dot in any action that you feel reflects your character’s background.
\item Assign two more dots anywhere you please (max rating is 2, remember).
\end{itemize}

\subsection{Choose a Special Ability}

Take a look at the special abilities for your playbook and choose one. If you can’t decide which one to pick, go with the first one on the list—it’s placed there as a good default choice.

\subsubsection{Special Armor}

Some special abilities refer to your \textbf{special armor}. Each character sheet has a set of three boxes to track usage of armor (standard, heavy, and special). If you have any abilities that use your special armor, tick its box when you activate one of them. If you don’t have any special abilities that use special armor, then you can’t use that armor box at all.

\subsection{Choose One Close Friend and One Rival}

Each playbook has a list of NPCs that your character knows. Choose one from the list who is a close relationship (a good friend, a lover, a family relation, etc.). Mark the upward-pointing triangle next to their name. Then choose another NPC on the list who’s your rival or enemy. Mark the downward-pointing triangle next to their name.

\subsection{Choose Your Vice}

Every character is in thrall to some vice or another, which they indulge to deal with stress. Choose a vice from the list, and describe it on the line above with the specific details and the name and location of your \textbf{vice purveyor}.

\begin{description}
\item[Faith] You’re dedicated to an unseen power, forgotten god, ancestor, etc.
\item[Gambling] You crave games of chance, betting on sporting events, etc.
\item[Luxury] Expensive or ostentatious displays of opulence.
\item[Obligation] You’re devoted to a family, a cause, an organization, a charity, etc.
\item[Pleasure] Gratification from lovers, food, drink, drugs, art, theater, etc.
\item[Stupor] You seek oblivion in the abuse of drugs, drinking to excess, getting beaten to a pulp in the fighting pits, etc.
\item[Weird] You experiment with strange essences, consort with rogue spirits, observe bizarre rituals or taboos, etc.
\end{description}

\subsection{Record Your Name, Alias \& Look}

Choose a name for your character from the sample list, or create your own. If your character uses an alias or nickname in the underworld, make a note of it. Record a few evocative words that describe your character’s look (samples provided on the next page).

\subsection{Review your details}

Take a look at the details on your character sheet, especially the \kw{xp} \textbf{triggers} for your playbook (like ``Earn xp when you address a challenge with knowledge or arcane power,'' for example) and the \textbf{special items} available to a character of your type (like the Whisper’s spirit mask, for example). You begin with access to all of the items on your sheet, so don’t worry about picking specific things---you’ll decide what your character is carrying later on, when you’re on the job.

That’s it! Your character is ready for play. When you start the first session, the GM will ask you some questions about who you are, your outlook, or some past events. If you don’t know the answers, make some up. Or ask the other players for ideas.

\subsection{Character Creation Summary}

\begin{enumerate}
\item \textbf{Choose a playbook.} Your playbook represents your character’s reputation, their special abilities, and how they advance.
\item \textbf{Choose a heritage.} Detail your choice with a note about your family life. For example, Ore miners, now war refugees.
\item \textbf{Choose a background.} Detail your choice with your specific history. For example, Labor: Hunter, mutineer.
\item \textbf{Assign four action dots.} No action may begin with a rating higher than 2 during character creation. (After creation, action ratings may advance up to 3. When you unlock the Mastery advance for your crew, you can advance actions up to rating 4.)
\item \textbf{Choose a special ability.} They’re in the gray column in the middle of the character sheet. If you can’t decide, choose the first ability on the list. It’s placed there as a good first option.
\item \textbf{Choose a close friend and a rival.} Mark the one who is a close friend, long-time ally, family relation, or lover (the upward-pointing triangle). Mark one who is a rival, enemy, scorned lover, betrayed partner, etc. (the downward-pointing triangle).
\item \textbf{Choose your vice.} Pick your preferred type of vice, detail it with a short description and indicate the name and location of your vice purveyor.
\item \textbf{Record your name, alias, and look.} Choose a name, an alias (if you use one), and jot down a few words to describe your look. Examples are provided on the preceding page.
\end{enumerate}

\subsection{Loadout}

You have access to all of the \textbf{items} on your character sheet. For each operation, decide what your character’s \kw{Load} will be. During the operation, you may say that your character has an item on hand by checking the box for the item you want to use—up to a number of items equal to your chosen \kw{Load}. It also determines your movement speed and conspicuousness:

\begin{description}
\item[1-3 \kw{Load}] Light. You’re faster, less conspicuous; you blend in with citizens.
\item[4-5 \kw{Load}] Normal. You look like a scoundrel, ready for trouble.
\item[\hspace{0.83em}6 \kw{Load}] Heavy. You’re slower. You look like an operative on a mission.
\item[7-9 \kw{Load}] Encumbered. You’re overburdened and can’t do anything except move very slowly.
\end{description}

Some special abilities (like the \kw{Mule} ability or Assassin’s Rigging) increase the load limits.

Some items count as two items for load (they have two connected boxes). \textit{Items in italics don’t count toward your load.}

You don’t need to select specific items now. Review your personal items and the standard item descriptions.
