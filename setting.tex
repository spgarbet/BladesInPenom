\chapter{Introduction}


\section{Eye of Opening the Way}

Beneath a fog of insects, beside rolling black waters lapping against dun colored mud, Penóm peaks up above the hot fog on three parallel ridges of Bedrock. Great ocean vessels here exchange their cargo with barges heading up the Turina river bound for Tumíssa or even further upstream. Beholden to Tsolyánu, the Empire of the Petal Throne, the city maintains a veneer of civilization over the rot and moral decay of it's citizens. The emperor may see glowing reports on but the truth is vastly different. The black market here is the principal market and business is mostly by a private greasing of palms to smooth the way. Decay pervades all dimensions of life in Penóm.

In \emph{Blades in Penóm}, you take on the role of \textbf{clan members} struggling to carve out honor and recognition for your clan. Your clan once dominated Penóm, but assassinations and  treachery has reduced your holdings to near bankrupcy. You've taken control from the clan elders, who have died in mysterious circumstance and seek to rebuild your family name into something to fear.

Who knows what really motivates the strange non-humans, demons, or gods and where the source of magic stems from? What's important is in this stew, a savvy band might be able to make just the right opportunities to upset the apple cart and bring honor to your ancestors.

\subsection{The World}

Tékumel is a high fantasy role play setting created by M.A.R. Barker. It's history and setting span a rough 25,000 years and before that was the Time of Darkness.  The world of Tekumel was once a great hub of intergalactic trade. For reasons unknown, it's solar system was ripped from normal space time and it spins alone in a bubble universe. Some think it was hostile action, and others think it was overuse of faster than light travel. The ensuing chaos resulted in a long Time of Darkness, as Tékumel is a world poor in heavy metals and other resources necessary to sustain an advanced technological society. The thin nature of the fabric of reality in this pocket universe allows for magic, direct manipulation by nearby \emph{n}-dimensional beings "demons", and other singular powerful extra-dimensional beings "gods".

Tékumel has layers upon layers of collapsed civilizations from technologically advanced societies to alien, all built upon the ruins of the previous. What lies beneath it all is still unknown. 


\subsection{What is this Game About?}

\begin{itemize}
\item Coming together to work as a team, building alliances, negociating deals to bring honor to your clan.
\item Hosting noble dinner parties while simulataneously using subterfuge and thievery to advance your mission.
\item Understanding the web that makes a city function.
\item Taking back power from a corrupt elite. Putting fear into those who stand in your way.
\end{itemize}

\subsubsection{Dramatic Questions}

\begin{itemize}
\item If noone ever finds out, is an action ignoble?
\item How does one maintain civility and manners while secretely at war?
\item It's easy to destroy, but what does it take to build?
\end{itemize}

The Game

Blades in the Dark is a game about a group of daring characters building an enterprising crew. We play to find out if the fledgling crew can thrive amidst the teeming threats that surround it.
The Players

Each player creates a character and works with the other players to create the crew to which their characters belong. Each player strives to bring their character to life as an interesting, daring character who reaches boldly beyond their current safety and means.

The players work together with the Game Master to establish the tone and style of the game by making judgment calls about the mechanics, dice, and consequences of actions. The players take responsibility as co-authors of the game with the GM.
The Characters

The characters attempt to develop their crew by performing scores and contending with threats from their enemies.
The Crew

In addition to creating characters, you’ll also create the crew by choosing which type of criminal enterprise you’re interested in exploring.
The Game Master

The GM establishes the dynamic world around the characters. The GM plays all the non-player characters in the world by giving each one a concrete desire and preferred method of action.

The GM helps organize the conversation of the game so it’s pointed toward the interesting elements of play. The GM isn’t in charge of the story and doesn’t have to plan events ahead of time. They present interesting opportunities to the players, then follow the chain of action and consequences wherever they lead.
Playing A Session

A session of Blades in the Dark is like an episode of a TV show. There are one or two main events, plus maybe some side-story elements, which all fit into an ongoing series. A session of play can last anywhere from two to six hours, depending on the preferences of the group.

During a session, the crew of scoundrels works together to choose a score to accomplish, then they make a few dice rolls to jump into the action of the score in progress. The PCs take actions, suffer consequences, and finish the operation (succeed or fail). Then the crew has downtime, during which they recover, pursue side-projects, and indulge their vices. After downtime, the players once again look for a new opportunity or create their own goals and pursuits, and we play to find out what happens next.


\section{Eye of Illuminating Glory}

\subsection{Tékumel}

Today there are five great empires with a rigid caste system and complex hierarchies of social interactions. Many of the devices of the ancients still function, and found working technologies are highly prized. There are five principal human empires: Tsolyánu, Livyánu, Mu’ugalavyá, Yán Kór, and Salarvyá.

The Empire of the Petal Throne, Tsolyánu, ruled by the Emperor in the Golden Tower in Avanthár and where the city of Penóm resides. The power of the Emperor is absolute, served by the ever-vigilant eyes of the Omnipotent Azure Legion and its agents. It's citizens are coppery brown to golden tan with aquiline noses, no body hair and fine straight black hair. Deviations from this are considered ugly. The other empires view the Tsolyáni as arrogant, officious, and overrefined, a nation always striving to live up to the unattainable standard of its Engsvanyáli ancestors.

A mysterious enigma wrapped in deep sorcery known as Livyánu lies just southwest of Tsolyánu. The Livyáni are a tall and graceful people with slender builds. The ruling theocracy of the Shadow Gods and it's workings are never revealed to outsiders. Most ranks and status of Livyáni citizens is deliberately concealed and only known to inner initiates of each circle of the priesthoods. It is said the intricate facial tattoos contain all the details if one only knew how to read them. The one who seems to speak with most authority externally is ásqar Gyardánaz, ‘Principal Staff of the Glory of Qame’él’, and it is said he presides over the central Council of the Priesthoods. The summoning of 'Demons' as a large part of their rituals is known for certain.

Mu’ugalavyá west of Tsolyánu and it's greatest rival. Viewed as blockheads by the Tsolyáni, but also stiffly correct, honest and 'noble' in action. They are a bit shorter with rather flat hook noses, greyish eyes and straight black hair. Their religion and customs are similar to that of Tsolyánu. Their land is split into four provices each ruled by a 'Palace'. The four princes of these palaces form the central government, the Four Palaces of the Square. There is an active standoff along the border of the Cháka Protectorates between the red lacquered troops of Mu’ugalavyá and the blue-clad Tsolyáni.

Yán Kór to the north is a hardy land of sturdy tribes with a matriarchal society. It's current leader, Baron áld, has vowed revenge upon Tsolyánu for the death of his lover Yilrána at the hands of Tsolyáni troops. He has banded together the various city-states of the north and is actively waging war. 

Salarvyyá a feudal government with seven dynasties dividing the land in to be farmed by lesser clans in a complex system of fiefdoms. Each of the seven families holds one or more monopoly: the Chruggilléshmu family of Tsatsayágga has charge of the sea and shipping along the southern coast and also holds the reins of the central government; the Hrüchcháqsha family of Chamé’el has a monopoly upon the Vrélq, the crustacean which produces the black dye used for clothing and dyeing armour; etc. The Salarvyáni are quite distinctive: of the same average height as the Tsolyáni, but with more sallow complexions of almost a pale yellowish tan, and they are generally more hirsute, with heavy body hair, curly or even kinky beards and sickleshaped noses. They tend towards obesity, especially after age 30. The Tsolyáni regard them as feudal hotheads, ‘greasy men with beards like woven rugs’, who have nothing better to do than squabble over trifles.



\subsection{Penóm: Eye of Elaborate Putrefaction}

Penóm is a silken perfumed carbuncle on the southern coast of Tsolyánu. Its position makes it a  major port of trade for a wealth of goods to and from places like Salarvya and Livyanu to the interior of Tsolyánu. Grand palaces and temples sit upon it's three ridges of bedrock filled with mannerly well dressed nobles amidst a hell-hole of rot, decay, disease and bugs. Such riches stirred together with desperation and misery make for an unstable stew. Tsolyánu in general is incredibly harsh on crime and just the risk of loss of status alone is a major deterrent, but as the old aphorism goes, a man of low clan will be impaled for stealing a glass of chumetl(buttermilk) but a man of noble clan will be celebrated for stealing a province. Thus it goes in Penóm, where stealing a province is perfected to an almost respectable high art. The majority of black market goods in Tsolyánu pass through Penóm hands at some point. This trade is done by a very mannerly veneer of polite fictions spun around the real skein of power. A noble might remark 'Recent acquisitions were costlier than expected', which really means 'I didn't expect so much killing on that heist'. The Empire of the Petal Throne has lost many an agent attempting to pierce the veil of secrecy surrounding trade in Penóm. A retributive Ditlana (a razing and rebuilding) has been threatened by the Emperor but key individuals always meet sudden ends at just the right moment. In Penóm, the unspoken highest rule is to never talk at all of such deep business. At best it is insulting and vulgar. At worst one is quickly sacrificed to a dark god. Further complicating this dynamic situation are all the foreign agents, non human communities, conspiracies and cults of the pariah gods (the One Other, the One Who Is, the Goddess of the Pale Bone, the One of Fears, and the Mad One of Hl'kku). The complexity and difficulty of life in Penóm cannot be overstated. Yet locals call Penóm, 'The Untrying Weave', and approach life in a lackadaisical manner not seen anywhere in Tekumel.

Penóm was physically much higher and mountainous in the far distance past but as a result of the Engsvanyali catastrophe, it sank till it is surrounded by swamps. Dams and dikes without allowing for redeposition of silt have caused some areas in the city to sink below sea level! These areas are of course the slums. Curiously through devices of the ancients there is a fresh water supply and a complex of pump systems still functioning. This poorly understood machinery of the ancients has saved the city from several terrible storms, but not the slums which get mostly washed away about every twenty years.

Following the Engsvanyali was the ``Time of Chaos'' in which arose one legendary Hagarr of Paranta (southeast of Penóm). It is said he was transporting goods to market for his clan when his boat was beset by pirates. His brother was killed in the attack, and this enraged him and he slew all ten pirates with his bare hands. The story of this feat spread quickly among sailors, and he was soon leading fighting forces in a metal-hulled ship against the Hlyss, the Hlutrgu, and any sea monster unlucky enough to cross his path. In the end he "sailed his great ship into the Islands of the Sky, where the Lords of Many Lights there did him homage." Some disputed accounts say his expertise was needed by the College at the End of Time and he will return one day in Penóm's darkest hour.

By decree in 2251 A.S. Arshu'u ``the Ever-Splendid'' ordered the establishment of a marine force to protect trade due to continued pirate activity and Hlyss attacks. The Red Sky Clan which claims to be the direct descents of Hagarr was awarded the honor of commanding the 1st Imperial Marines; The Flotilla of Hagarr of Paranta. The agreement was that they would recruit troops and pay for their upkeep if Imperium paid for the ships. The marines "patrolled" as far as Livyanu and Tsatsayagga in Salarvya. They delivered and received large ``samples'' of goods thus assisting in the Empire's commerce. Arshu'u's court found this utterly scandalous, and had seven or eight captains impaled with their crews. The 1st Imperial Navy does not allow officers from the Clan of the Red Sky after this incident. The current commander, Hagarr hiChunmiyel, is an old seasoned sailor who also owns a small fleet of merchant ships. His son, Miridame hiChumiyel, is being groomed to take command.

Emperor Dhich'une's general, Talrek hiKúrodu, is raising a legion in the southern marshlands just outside of Penóm. The rumors are that there are at least three cohorts of shedra (undead) present in it. The concordat prohibits the use of such troops. Such rumors have been spread by Dhich'une's rivals before.

\subsubsection{Cultures}

Clan identity is foremost in all culture throughout Tsolyanu. The caste system is very rigid, with huge expectation put upon behavior from those in higher clans. Penóm additionally has a mix of cultures from all over the world represented. From the mysterious tattooed Livyani, to the stoic Salarvyani, the rough and humorless Mu'ugalavyani. Numerous enclaves of nonhumans exist as well. Penóm is a melting pot of foreign interests with regular infusions of foreign merchants and sailors. The Omnipotent Azure Legion has taken to treating Penóm as more of a containment operation than being active against foreign agents.  However, they will not hesitate to kill anyone they determine to be acting against the empire's interests.

There are finishing schools and consultants available to help navigate the intricate maze of customs. They command a high prices, and without their assistance a party or reception will quickly become a social disgrace.

\subsubsection{Languages}

Tsolyáni is the principal language of Penóm. Likewise the languages of the other four human empires can be heard in the right sections of Penóm. Engsvanyáli is the older root language of the prior major human empire. Sunúz is of interest because, although it is obscure, it is quite useful for sorcerous purposes. For instance, Sunúz contains terms to describe movement in a six dimensional multi-planar space, something of use to beings who visit the other planar realms where ``demons'' live.

\subsubsection{Energy of the Planes}

The multi-planar energies leak and are channelled and can vary greatly from location to location. In some areas sorcery fails to work, and in others it is wildly successful. A wise sorcerer will test the local fabric of space gently before unleashing a major fireball, worse than it sputtering out it could possibly start a conflagration which burns everything nearby including the caster to atomic dust. There are many higher forces involved in complex tug of war trying to restore Tekumel to normal space, those trying to keep it in a bubble, and those trying to destroy it completely. Where these forces are in play nearby the thinness of the skein of reality can shift unexpectedly.

\subsubsection{Weather, Calendar, \& Seasons}

The weather in Penóm is tropical and hot. The majority of the year finds Penóm enveloped in a thick soup of humid air that sucks the life out of any outdoor activity. It gets worse as the day goes on till around 4 in the afternoon a sudden heavy downpour occurs. This is not a time to rejoice, because the rain  sizzles off the hot stones and forms a sauna of low hanging steamy fog till it cools and condenses after sunset. Penóm's reputation as a night city is for practical reasons. The city blossoms into activity the moment the sun has set.

There are occasional tropical depressions which move in and cool the whole region, all while dumping torrential rains upon the city. Some walkways through the low areas between the ridges are designed to float as the water levels rise in these areas. The winter period is a 4 week monsoon season before returning back to it's usual pattern of humidity, heat and fog.

There are also some regular events throughout the year. The Rain of Sárku occurs in spring, in which for a day it rains small poisonous purple worms from the sky. These back on the hot stones of the streets the next day and form hot odorous slicks--watch your step! There are also blooms of various biting insects which appear initially as dark clouds on the horizon and residents need to shelter from them. Weather forecast in Penóm are important as priests watch the various biological cycles to keep the citizens prepared.

\begin{description}[font=\sffamily\bfseries, leftmargin=1cm]
\item[Hasanpor] Monsoon Season; torrential rains
\item[Shapru]   Unfolding Twilight of Civility; a brief interlude of near pleasant weather.
\item[Didom]    Tide of Flies; clouds of newly hatched insects.
\item[Langala]  Rain of Sárku; a day in which poisonous worms rain from the sky.
\item[Fesru]    Heavenly Blossoms; colorful petals blow in the wind (poisonous).
\item[Drenggar] Arising of Contemplation; heat is unbearable during day, plants wilt.
\item[Firasul]  The Burning Stillness; hot with a lack of trade winds.
\item[Pardan]   When Demons Roost; height of tropical storm season. Sudden unpredictable deluges. 
\item[Halir]    Obscuring Veil of Spores; fungal spores darken the ground and walls.
\item[Trantor]  Wheels in Motion; trade winds are strong, harvests are in.
\item[Lesdrim]  The Ever Restless Phlegm; colorful slimes begin migrating.
\item[Dohala]   Days of Drá; thick fogs obscure vision and dampness pervades all.
\end{description}

\subsubsection{Dining}

Wines and reddish-purple Dlel-fruit from the Hayuri Isle abound. Fish, shellfish, kelp-like plants are served deeply spiced with a lot of rendered animal fat. Stews of  Luó, squash-beetles, big soft gooey things with legs and feelers are used to thicken gumbos. Yáfa-rice and beans are commonly eaten for lunch on wash-day (Monday). 



===The Common Folk

===The Well-Off

===The Wealthy

===Nonhuman

==Law \& Order

==The Underworld

==Priesthood

==Military

==The Undercity

=Penom Map


==Landmarks

1 Bamesu Bay
2 Chaigavra River
3 Eye of Gires
4 Temples of Methunel Hill
5 Crown of Hruggar Ridge
6 Four Mansions on Kolumkan Ridge
7 Governor's Palace

==Districts

===Gires Marine Base
Just north of town is a large sprawling marine base full of sailors. They practice various drills and patrol the coast for pirates and Hlyss incursions.


===Pavar's Tsan
A bustling row of temples and religious buildings sitting high on Metlunel Hill on the centeral ridge. Karakan has a well funded temple popular with sailors built from the remains of an earlier fortress. It features a prominent eternal flame out front. Ksarul's temple is built to look like series of ornate nested safes that are actually functional defenses, each must be passed through knowing the password of the day. Dlamelish is well situated as well in a city so full of vice.

===Resplendent Parchment District
A complex of ornate pyramidal buildings filled with humorless bureaucrats fills the southern ridge. God help the poor soul who tries to get anything done here. There is a spectacular monument in the center called the Dome of the Weeping Sea dedicated to Hruggar. At the end of Kolumkan Ridge is the great Eye of Gires a lighthouse to guide ships. Next to this on the steep craggy peninsula is the Crown of Hrugga, the governor's palace.

* Dremel hiSukosh of the Domed Tomb, obnoxious entitled examiner who is trying to climb the ranks. He was ordered to crack the Zu'ur trade and report back. In reality, he was posted here on the hope that Penom would make short work of him.
* The Governor of Penom, Lord Tiktikánu hiSSaronél, Clan of the Ripened Sheaf. An aristocratic, sophisticated man in his fifties. He appears effortless at minding and managing all the customs of his station. Underneath this facade is a man who views just about everyone as a threat. He turns a great profit from trading slaves to the Hlüss for Zu'úr, that is in no direct manner traceable to him. He tries to keep the Zu'úr shipments heading out of town, but inevitably some of the wares end up on the streets of Penom.
* Lord Chernáru hiSSaronél, the Governor's son. A petulant spoiled youth of twenty. Always aiming for a fight, and fancies himself an assassin in the circles of the Association of the Relievers from Life.
* Lady Shreku'él hiSSaronél, a striking beauty with deep curiosity and a Zu'úr habit.

===Ebbing of the Night District
Due to the swampy ground and limited bedrock, the bed here are not buried but exist in an above ground necropolis built mostly of locally quarried marble. Visitors in late evening and after dark are recommended to bring personal security forces with them due to the nearby slum. The tomb of hiVou is visited frequently by his former followers, it is decorated daily with copius offers of blessings. It is said that during when Gayél is new he will speak to the chosen of his followers with proper sacrifice.

===Desire's Heights
A sprawling shanty town full of narrow unplanned dead end alleys abutting against the nearby necropolis. It's built on the lowest ground in the city. Woe to any sailor who wanders this far in from the harbour. Insect netting here is prized as one of the most valuable possessions. The local city militia will not enter this district and being given guard duty here is prized as paid time off. The locals have a complex and ever changing set of agreements over what is permissible and not.

* Hrgz Ggrgz the Shén. An ignoble and untrustworthy leader of a band of thieves makes his home here. He does no direct business in Desire's Heights, but rather uses it as a base of operations and recruitment.

===Barbarian's Rest
A maze-like series of enclaves catering to non-citizens, Ahoggya, Shen, Pachi Lei, Pe Choi, Tinaliya, Pygmy Folk, and Hlaka. Each species architecture is jarring abutted up against their neighbors. The Shen and Ahoggya have settled at opposite ends due to their strong enmity. The Livyani restaurants are celebrated and well worth a visit.

===Four Mansions (Wealthy)
The northern ridge with outstanding views of the Bamesa Bay and overlooking the harbor. It is filled with the very highest clans, and there are four palaces that claim this space. Each is almost a minor town unto itself. Security here is very high. It features in a central park the Garden of Amber Twilight, with tropical species collected from around the world. Admission is free for residents of the Four Mansions district, but requires an increasingly sizable donation as ones status sinks lower. 

===Tsechelnu Flats View (Middle)
The middle status clans have several well constructed dwellings along the inner part of the northern ridge with not so flattering views of the Flats of Tsechelnu. The high point is a curious half sphere dome called Stone Hill. In it's center is a mysterious indentation. It is said to control the fate of the city via a gem which fits in the indentation.
  
===Daylight Ballet (Market)
This market caters to all interests and is next to the Warehouse. There are long colonnades with booths  between them. The smells, and colors and odors are overwhelming and deals are brokered in every direction. There is a pecking order for placement of each vendor and the premium spot has easy access to the Four Mansions district. The Bounty of Unspeakables, the night market, held under green light here features truly unusual delights and truly unique wares are brought out when either moon Gayél or Káshi is new.

* Rikam hiGaragu, the old caravan-master. Very knowledgable about surviving in the swamps, and makes a sizable sum for his caravan guidance. He never asks of cargo, nor pries into others business.


===Warehouse (Harbour)

Convient to the harbour for loading and unloading and full of canals granting easy flat barge movement of goods in and out. Most buildings here are three stories and the ornamentation feels oppressive as it looms overhead. This is by design as precious cargo is stored on upper floors via ropes and pulleys which descend from the overhead stonework. Clan members live on the first floors to protect the trade goods.

==Factions

===Underworld

* The Hive (IV): A Guild of Merchants who secretly trade in contraband. Named for their symbol, a golden Bee. Their leader is known only as "The Eel"
* Ndálu Clan(IV) An insidious criminal enterprise with secret membership in the ranks of Ksárul worshipers. Thought to pull the strings of the entire underworld. 
* The Circle of Flame (III) A secret society of antiquarians and scholars, cover for extorsion, graft, vice, and murder.
* Lord Scur'hla (III) An ancient noble said to be immortal, a follower of Sarku and most likely undead. Obsessed with arcane secrets.
* The Silver Nails (III) A company of mercenaries turned to crime when the war for Pan Chaka ended. Renowned Ssu slayers. 
* The Billhooks (II) A tough gang of thugs wielding hatchets and meat hooks.
* The Crows (II) An old gang with new leadership. Know for running illegal games of chance and extorsion rackets. 
* The Dimmer Sisters(II) House bound recluses with an occult reputation.
* The Gray Cloaks (II) Former bluecoats who turned to crime.
* The Grinders (II) A vicious gang of former dockers and leviathan blood refinery workers from Skovlan.
** The Lampblacks (II) The former lamp-lighter guild, ...
* The Red Sashes (II) Originally an XXX school of swordsmanship, expanded into criminal endeavors.
* The Wraiths (II) A mysterious crew of masked thieves and spies.
* The Fog Hounds (I) A crew of smugglers looking for a patron.
* The Lost (I) A group of street-toughs and ex-soldiers dedicated to protecting the downtrodden and the hopeless.
* "Loud Belly" (I) A brutal adolescent Ahoggyá, newly arrived in Penom from Onmu Tle Hlektis, fighting everyone for turf. His real name is unpronounceable and is said to sound like sewage going down a drain.

===Fringe

* The Palace of the Priesthoods of the Gods (IV) The ruling body between the temples of Pavar. Very slow to move, very harsh in response. Their inner membership is secret. They work in the palace of the priesthoods of the gods.
* The Pariah Gods (III) Those who worship the pariah gods or demons. There are many cults, who rarely organize together. An individual cult is usually Tier I or Tier II.
* Brotherhood of the Victory of the Worm (III) Undead who walk among the living controlled by an unknown force for an unknown purpose. Worshipers of Sarku.
* Society of the Blue Light (III) A secret group of Ksárul worshipers which gathers knowledge for its personal and collective power.

** The Reconciled (III) feral sprits ?
* Salarvya Refugees (III) Those fleeing the city-state wars consuming the lands to the East of Penom. Forced into criminal opportunities.
* Talrek hiKúrodu Legion of Shedra (II)
* The Dancing Maidens of Temptation (II) Dlamelish worshipers control the brothels, drug and vice trade inside the city of Penom. Always working to entice fresh upcoming recruits into the their ranks.

===Institutions
* The Flotilla of Hagarr of Paranta (VI) The armed forces of the 1st Imperial Marines. Garrisons are posted at the Governor's stronghold, aboard a trireme destroyer, at the outer gate for about 250 troops in total.
* Palace of the Realm (V) The government office in charge of all domestic affairs, public works, taxes, the judicial system, trade, transport, the necropolises and labor.
** Leviathan Hunters (V)
* Pure Light Society (V) Hnalla worshippers who seek stability above all, especially the stability of their profitable enterprises. Their roots go back to the founding of Penom. 
* Hirilakte Arena (IV) An organized group who is quite adept at providing a circus show of ritual combat with crooked gambling.
*   (IV)
* Clan of Dark Water (IV) The local high clan of assassins and followers of Hru'u.
* Boys in Blue (III) City Militia. 
* The Palace of Foreign Lands (III). Government offices in charge of foreign relations, foreign trade and shipping, customs, diplomacy.
* Livyanu Consulate (III)
* Salarvya Consulate (III)
* Omnipotent Azure Legion (II) These are the highest ranking members of the Imperial Legions, but they have been unable to pierce the secrecy of Penom and thus their Tier ranking is quite low compared to it's standing in any other city of Tsolyanu.
* Pan Chaka Consoulate (I)
* Shenyu Consulate (I)

===Labor \& Trade (Clans)
* Clan of Sea Blue (IV) Very high clan descended from the royal families of the Bednalljans and correspondingly proud and arrogant. 
* Turning Wheel (III) Very low clan of carters, suttlers, wheelwrights, and transporters of goods. 
* Green Malachite (III) Medium clan of farmers, sailors, fisherman, etc.; mostly devoted to the Lords of Stability; based in Penom but has houses all along the souther coast and as far north as Usenanu on the Mssuma River. 
* Red Eye of Dawn (III) Medium clan of most renowned jewelers' in the Empire; largely made up of worshipers of Avanthe and Dilinala in spite of the "red" of its name; centered in Bey Su.
* Red Flower (III) Medium clan descended from old Vrayani merchants with sailing, shipping and foreign trading interests; mostly devoted to Karakan and Chegarra. Centered on the island of Vra.
* Copper Door (II) Medium clan of merchants and moneylenders; usually followers of Sarku. 
* Flowering Life (II) Low clan of Rope and net-makers, fisherman, and shellfish gatherers; the majority is devoted to Avanthe and Dilinala, but in Penom almost exclusively Hnalla.     
* Purple Gem (II) High clan of official scribes and bookmaking clan.
* Sinking Land (II) Very low clan of peasants, producers of swamp products and fishterman; devoted Belkhanu and his Cohort Qon based in the lowlands around Penom. 
* Black Earth (II) Peasants, servants, artisans, fisherman, and swamp workers; followers of Hru'u and Wuru although other sects are found as well; based at Purdimal but found here in Penom as well.
* First Moon / Moon of Evening (I) These two medium status clans contain merchants and artisans, plus a few bureaucrats and priests; no religious affliation.

