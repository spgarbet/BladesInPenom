\chapter{The Rules}

\section{The Core System}

\subsection{Judgment calls}

When you play, you’ll make several key judgment calls. Everyone contributes, but either the players or the GM gets final say for each:

\begin{itemize}
\item Which actions are reasonable as a solution to a problem? Can this person be swayed? Must we get out the tools and tinker with this old rusty lock, or could it also be quietly finessed? The players have final say.
\item How dangerous and how effective is a given action in this circumstance? How risky is this? Can this person be swayed very little or a whole lot? The GM has final say.
\item Which consequences are inflicted to manifest the dangers in a given circumstance? Does this fall from the roof break your leg? Do the constables merely become suspicious or do they already have you trapped? The GM has final say.
\item Does this situation call for a dice roll, and which one? Is your character in position to make an action roll or must they first make a resistance roll to gain initiative? The GM has final say.
\item Which events in the story match the experience triggers for character and crew advancement? Did you express your character’s beliefs, drives, heritage, or background? You tell us. The players have final say.
\end{itemize}

\subsection{Rolling the Dice}

Blades in the Dark uses six-sided dice. You roll several at once and read the \textbf{single highest result}.

\begin{itemize}
\item If the highest die is a \textbf{6}, it’s a \textbf{full success}—things go well. If you roll more than one \textbf{6}, it’s a \textbf{critical success}—you gain some additional advantage.
\item If the highest die is a \textbf{4 or 5}, that’s a \textbf{partial success}—you do what you were trying to do, but there are consequences: trouble, harm, reduced effect, etc.
\item If the highest die is \textbf{1-3}, it’s a \textbf{bad outcome}. Things go poorly. You probably don’t achieve your goal and you suffer complications, too.
\end{itemize}

\begin{qb}
If you ever need to roll but you have zero (or negative) dice, roll two dice and take the single lowest result. You can’t roll a \textbf{critical} when you have zero dice.
\end{qb}

All the dice systems in the game are expressions of this basic format. When you’re first learning the game, you can always ``collapse'' back down to a simple roll to judge how things go. Look up the exact rule later when you have time.

To create a dice pool for a roll, you’ll use a \textbf{trait} (like your \kw{Finesse} or your \kw{Prowess} or your crew’s Tier) and take dice equal to its \textbf{rating}. You’ll usually end up with one to four dice. Even one die is pretty good in this game--a 50\% chance of success. The most common traits you’ll use are the action ratings of the player characters. A player might roll dice for their \kw{Skirmish} action rating when they fight an enemy, for example.

There are four types of rolls that you’ll use most often in the game:

\begin{description}[font=\rmfamily\bfseries\scshape, leftmargin=1cm]
\item[Action roll] When a PC attempts an action that’s dangerous or troublesome, you make an action roll to find out how it goes. Action rolls and their effects and consequences drive most of the game.
\item[Downtime roll] When the PCs are at their leisure after a job, they can perform downtime activities in relative safety. You make downtime rolls to see how much they get done.
\item[Fortune roll] The GM can make a fortune roll to disclaim decision making and leave something up to chance. How loyal is an NPC? How much does the plague spread? How much evidence is burned before the constables kick in the door?
\item[Resistance roll] A player can make a resistance roll when their character suffers a consequence they don’t like. The roll tells us how much stress their character suffers to reduce the severity of a consequence. When you resist that ``Broken Leg'' harm, you take some stress and now it’s only a ``Sprained Ankle'' instead.
\end{description}

\subsection{The Game Structure}

Blades in the Dark has a structure to play, with four parts. By default, the game is in \textbf{free play}—characters talk to each other, they go places, they do things, they make rolls as needed.

When the group is ready, they choose a target for their next operation, then choose a type of plan to employ. This triggers the \emph{engagement roll} (which establishes the situation as the operation starts) and then the game shifts into the \textbf{score} phase.

During the score, the PCs engage the target—they make rolls, overcome obstacles, call for flashbacks, and complete the operation (successfully or not). When the score is finished, the game shifts into the \textbf{downtime} phase.

During the downtime phase, the GM engages the systems for \emph{payoff}, \emph{heat}, and \emph{entanglements}, to determine all the fallout from the score. Then the PCs each get their \emph{downtime activities}, such as indulging their vice to remove stress or working on a long-term project. When all the downtime activities are complete, the game returns to \textbf{free play} and the cycle starts over again.

The phases are a conceptual model to help you organize the game. They’re not meant to be rigid structures that restrict your options (this is why they’re presented as amorphous blobs of ink without hard edges). Think of the phases as a menu of options to fit whatever it is you’re trying to accomplish in play. Each phase suits a different goal.

\section{Actions \& Attributes}

\subsection{Action Ratings}

There are 12 \textbf{actions} in the game that the player characters use to overcome obstacles.

\begin{item3}
\item Attune
\item Command
\item Consort
\item Finesse
\item Hunt
\item Prowl
\item Skirmish
\item Study
\item Survey
\item Sway
\item Tinker
\item Wreck
\end{item3}

Each action has a rating (from zero to 4) that tells you how many dice to roll when you perform that action. Action ratings don’t just represent skill or training—you’re free to describe how your character performs that action based on the type of person they are. Maybe your character is good at \kw{Command} because they have a scary stillness to them, while another character barks orders and intimidates people with their military bearing.

You choose which action to perform to overcome an obstacle, by describing what your character does. Actions that are poorly suited to the situation may be less effective and may put the character in more danger, but they can still be attempted. Usually, when you perform an action, you’ll make an \textbf{action roll} to see how it turns out.

\subsection{Action Roll}

You make an \textbf{action roll} when your character does something potentially dangerous or troublesome. The possible results of the action roll depend on your character’s \textbf{position}. There are three positions: \textbf{controlled}, \textbf{risky}, and textbf{desperate}. If you’re in a \textbf{controlled} position, the possible consequences are less serious. If you’re in a \textbf{desperate} position, the consequences can be severe. If you’re somewhere in between, it’s \textbf{risky}--usually considered the ``default'' position for most actions.

If there’s no danger or trouble at hand, you don’t make an action roll. You might make a \textbf{fortune} roll or a textbf{downtime} roll or the GM will simply say yes—and you accomplish your goal.

\subsection{Attribute Ratings}

There are three \textbf{attributes} in the game system that the player characters use to resist bad consequences: \kw{Insight}, \kw{Prowess}, and \kw{Resolve}. Each attribute has a rating (from zero to 4) that tells you how many dice to roll when you use that attribute.

The rating for each attribute is equal to the number of dots in the \textbf{first column} under that attribute (see the examples, at right). The more well-rounded your character is with a particular set of actions, the better their attribute rating.

\subsection{Resistance Roll}

Each attribute resists a different type of danger. If you get stabbed, for example, you resist physical harm with your \kw{Prowess} rating. Resistance rolls always succeed—you diminish or deflect the bad result—but the better your roll, the less \textbf{stress} it costs to reduce or avoid the danger.

When the enemy has a big advantage, you’ll need to make a resistance roll before you can take your own action. For example, when you duel the master sword-fighter, she disarms you before you can strike. You need to make a resistance roll to keep hold of your blade if you want to attack her. Or perhaps you face a powerful ghost and attempt to \kw{Attune} with it to control its actions. But before you can make your own roll, you must resist possession from the spirit.

The GM judges the threat level of the enemies and uses these ``preemptive'' resistance rolls as needed to reflect the capabilities of especially dangerous foes.

Find out more about Resistance Rolls in  \S Resistance \& Armor.

FIXME HERE
This character has a Hunt action rating of 1.
Their Insight attribute rating is 1 (the first column of dots). Insight
They also have Prowl 1 and Skirmish 2.
Their Prowess attribute rating is 2. Prowess

\subsection{Actions}

When you \kw{Attune}, you open your mind to arcane power.

\begin{qb}You might communicate with a ghost. You could try to perceive beyond sight in order to better understand your situation (but Surveying might be better).\end{qb}

When you \kw{Command}, you compel swift obedience.

\begin{qb}You might intimidate or threaten to get what you want. You might lead a gang in a group action. You could try to order people around to persuade them (but Consorting might be better).\end{qb}

When you \kw{Consort}, you socialize with friends and contacts.

\begin{qb}You might gain access to resources, information, people, or places. You might make a good impression or win someone over with your charm and style. You might make new friends or connect with your heritage or background. You could try to manipulate your friends with social pressure (but Sway might be better).\end{qb}

When you \kw{Finesse}, you employ dextrous manipulation or subtle misdirection.

\begin{qb}You might pick someone’s pocket. You might handle the controls of a vehicle or direct a mount. You might formally duel an opponent with graceful fighting arts. You could try to employ those arts in a chaotic melee (but Skirmishing might be better). You could try to pick a lock (but Tinkering might be better).\end{qb}

When you \kw{Hunt}, you carefully track a target.

\begin{qb}You might follow a target or discover their location. You might arrange an ambush. You might attack with precision shooting from a distance. You could try to bring your guns to bear in a melee (but Skirmishing might be better).\end{qb}

When you \kw{Prowl}, you traverse skillfully and quietly.

\begin{qb}You might sneak past a guard or hide in the shadows. You might run and leap across the rooftops. You might attack someone from hiding with a back-stab or blackjack. You could try to waylay a victim in the midst of battle (but Skirmishing might be better).\end{qb}

When you \kw{Skirmish}, you entangle a target in close combat so they can’t easily escape.

\begin{qb}You might brawl or wrestle with them. You might hack and slash. You might seize or hold a position in battle. You could try to fight in a formal duel (but Finessing might be better).\end{qb}

When you \kw{Study}, you scrutinize details and interpret evidence.

\begin{qb}You might gather information from documents, newspapers, and books. You might do research on an esoteric topic. You might closely analyze a person to detect lies or true feelings. You could try to examine events to understand a pressing situation (but Surveying might be better).\end{qb}

When you \kw{Survey}, you observe the situation and anticipate outcomes.

\begin{qb}You might spot telltale signs of trouble before it happens. You might uncover opportunities or weaknesses. You might detect a person’s motivations or intentions. You could try to spot a good ambush point (but Hunting might be better).\end{qb}

When you \kw{Sway}, you influence with guile, charm, or argument.

\begin{qb}You might lie convincingly. You might persuade someone to do what you want. You might argue a compelling case that leaves no clear rebuttal. You could try to trick people into affection or obedience (but Consorting or Commanding might be better).\end{qb}

When you \kw{Tinker}, you fiddle with devices and mechanisms.

\begin{qb}You might create a new gadget or alter an existing item. You might pick a lock or crack a safe. You might disable an alarm or trap. You might turn the clockwork devices around the city to your advantage. You could try to use your technical expertise to control a vehicle (but Finessing might be better).\end{qb}

When you \kw{Wreck}, you unleash savage force.

\begin{qb}You might smash down a door or wall with a sledgehammer, or use an explosive to do the same. You might employ chaos or sabotage to create a distraction or overcome an obstacle. You could try to overwhelm an enemy with sheer force in battle (but Skirmishing might be better).\end{qb}

As you can see, many actions overlap with others. This is by design. As a player, you get to choose which action you roll, by saying what your character does. Can you try to \kw{Wreck} someone during a fight? Sure! The GM tells you the position and effect level of your action in this circumstance. As it says, \kw{Skirmish} might be better (less risky or more effective), depending on the situation at hand (sometimes it won’t be better).

\section{Stress \& Trauma}

\subsection{Stress}

Player characters in Blades in the Dark have a special reserve called \textbf{stress}. When they suffer a consequence that they don’t want to accept, they can take stress instead. The result of the \textbf{resistance roll} determines how much stress it costs to avoid a bad outcome.

\begin{qb}During a knife fight, Daniel’s character, Cross, gets stabbed in the chest. Daniel rolls his \kw{Prowess} rating to resist, and gets a \textbf{2}. It costs 6 stress, minus 2 (the result of the resistance roll) to resist the consequences. Daniel marks off 4 stress and describes how Cross survives.\end{qb}

    The GM rules that the harm is reduced by the resistance roll, but not avoided entirely. Cross suffers level 2 harm (``Chest Wound'') instead of level 3 harm (``Punctured Lung'').

\subsection{Pushing Yourself}

You can use stress to push yourself for greater performance. For each bonus you choose below, take \textbf{2 stress} (each can be chosen once for a given action):

\begin{itemize}
\item Add \textbf{+1d} to your roll. (This may be used for an action roll or downtime roll or any other kind of roll where extra effort would help you)
\item Add \textbf{+1 level} to your effect.
\item Take action when you’re incapacitated.
\end{itemize}

\subsection{Trauma}

When a PC marks their last stress box, they suffer a level of \kw{Trauma}. When you take \kw{Trauma}, circle one of your \textbf{trauma conditions} like \emph{Cold, Reckless, Unstable,} etc. They’re all described below.

When you suffer trauma, you’re taken out of action. You’re ``left for dead'' or otherwise dropped out of the current conflict, only to come back later, shaken and drained. When you return, \textbf{you have zero stress} and your vice has been satisfied for the next downtime.

\textbf{Trauma conditions are permanent}. Your character acquires the new personality quirk indicated by the condition, and can earn xp by using it to cause trouble. \textbf{When you mark your fourth trauma condition}, your character cannot continue as a daring scoundrel. You must retire them to a different life or send them to prison to take the fall for the crew’s \kw{Wanted Level}.

\subsection{Trauma Conditions}

\begin{description}
\item[Cold] You’re not moved by emotional appeals or social bonds.
\item[Haunted] You’re often lost in reverie, reliving past horrors, seeing things.
\item[Obsessed] You’re enthralled by one thing: an activity, a person, an ideology.
\item[Paranoid] You imagine danger everywhere; you can’t trust others.
\item[Reckless] You have little regard for your own safety or best interests.
\item[Soft] You lose your edge; you become sentimental, passive, gentle.
\item[Unstable] Your emotional state is volatile. You can instantly rage, or fall into despair, act impulsively, or freeze up.
\item[Vicious] You seek out opportunities to hurt people, even for no good reason.
\end{description}

\section{Progress Clocks}

\begin{qb}Sneaking into the constables watch tower? Make a clock to track the alert level of the patrolling guards. When the PCs suffer consequences from partial successes or missed rolls, fill in segments on the clock until the alarm is raised.\end{qb}

\begin{wrapfigure}[15]{r}{0.2\textwidth}\begin{center}\includegraphics[scale=0.2]{img/progress-clocks.png}\end{center}\end{wrapfigure}

A \textbf{progress clock} is a circle divided into segments (see examples at right). Draw a progress clock when you need to track ongoing effort against an obstacle or the approach of impending trouble.


Generally, the more complex the problem, the more segments in the progress clock.

A complex obstacle is a 4-segment clock. A more complicated obstacle is a 6-clock. A daunting obstacle is an 8-segment clock.

When you create a clock, make it about the \textbf{obstacle}, not the method. The clocks for an infiltration should be ``Interior Patrols'' and ``The Tower,'' not ``Sneak Past the Guards'' or ``Climb the Tower.'' The patrols and the tower are the obstacles­—the PCs can attempt to overcome them in a variety of ways.

Complex enemy threats can be broken into several ``layers,'' each with its own progress clock. For example, the dockside gangs’ HQ might have a ``Perimeter Security'' clock, an ``Interior Guards'' clock, and an ``Office Security'' clock. The crew would have to make their way through all three layers to reach the gang boss’ personal safe and valuables within.

Remember that a clock tracks progress. It reflects the fictional situation, so the group can gauge how they’re doing. A clock is like a speedometer in a car. It shows the speed of the vehicle—it doesn’t determine the speed.

\subsection{Simple Obstacles}

Not every situation and obstacle requires a clock. Use clocks when a situation is complex or layered and you need to track something over time—otherwise, resolve the result of an action with a single roll.

\subsubsection{Danger Clocks}

The GM can use a clock to represent a progressive danger, like suspicion growing during a seduction, the proximity of pursuers in a chase, or the alert level of guards on patrol. In this case, when a complication occurs, the GM ticks one, two, or three segments on the clock, depending on the consequence level. When the clock is full, the danger comes to fruition—the guards hunt down the intruders, activate an alarm, release the hounds, etc.

\subsubsection{Racing Clocks}

Create two opposed clocks to represent a race. The PCs might have a progress clock called ``Escape'' while the constables have a clock called ``Cornered.'' If the PCs finish their clock before the constables fill theirs, they get away. Otherwise, they’re cornered and can’t flee. If both complete at the same time, the PCs escape to their lair, but the hunting officers are outside!

You can also use racing clocks for an environmental hazard. Maybe the PCs are trying to complete the ``Search'' clock to find the lockbox on the sinking ship before the GM fills the ``Sunk'' clock and the vessel goes down.

\subsubsection{Linked Clocks}

You can make a clock that unlocks another clock once it’s filled. For example, the GM might make a linked clock called ``Trapped'' after an ``Alert'' clock fills up. When you fight a veteran warrior, she might have a clock for her ``Defense'' and then a linked clock for ``Vulnerable.'' Once you overcome the ``Defense'' clock, then you can attempt to overcome the ``Vulnerable'' clock and defeat her. You might affect the ``Defense'' clock with violence in a knife-fight, or you lower her defense with deception if you have the opportunity. As always, the method of action is up to the players and the details of the fiction at hand.

\subsubsection{Mission Clocks}

The GM can make a clock for a time-sensitive mission, to represent the window of opportunity you have to complete it. If the countdown runs out, the mission is scrubbed or changes—the target escapes, the household wakes up for the day, etc.

\subsubsection{Tug-of-war Clocks}

You can make a clock that can be filled and emptied by events, to represent a back-and-forth situation. You might make a ``Revolution!'' clock that indicates when the refugees start to riot over poor treatment. Some events will tick the clock up and some will tick it down. Once it fills, the revolution begins. A tug-of-war clock is also perfect for an ongoing turf war between two crews or factions.

\subsection{Long-term Project}

Some projects will take a long time. A basic long-term project (like tinkering up a new feature for a device) is eight segments. Truly long-term projects (like creating a new designer drug) can be two, three, or even four clocks, representing all the phases of development, testing, and final completion. Add or subtract clocks depending on the details of the situation and complexity of the project.

A long-term project is a good catch-all for dealing with any unusual player goal, including things that circumvent or change elements of the mechanics or the setting.

\subsection{Faction Clocks}

Each faction has a long-term goal. When the PCs have \textbf{downtime}, the GM ticks forward the faction clocks that they’re interested in. In this way, the world around the PCs is dynamic and things happen that they’re not directly connected to, changing the overall situation in the city and creating new opportunities and challenges.

The PCs may also directly affect NPC faction clocks, based on the missions and scores they pull off. Discuss known faction projects that they might aid or interfere with, and also consider how a PC operation might affect the NPC clocks, whether the players intended it or not.

\section{Action Roll}


When a player character does something challenging, we make an \textbf{action roll} to see how it turns out. An action is challenging if there’s an obstacle to the PC’s goal that’s dangerous or troublesome in some way. We don’t make an action roll unless the PC is put to the test. If their action is something that we’d expect them to simply accomplish, then we don’t make an action roll.

\begin{qb}Each game group will have their own ideas about what ``challenging'' means. This is good! It’s something that establishes the tone and style of your Blades series.\end{qb}

To make an action roll, we go through six steps. In play, they flow together somewhat, but let’s break each one down here for clarity.

\begin{enumerate}
\item    The player states their \textbf{goal} for the action.
\item    The player chooses the \textbf{action rating}.
\item    The GM sets the \textbf{position} for the roll.
\item    The GM sets the \textbf{effect level} for the action.
\item    Add \textbf{bonus dice}.
\item    The player rolls the dice and we judge the result.
\end{enumerate}

\subsection{1. The Player States Their Goal}

Your goal is the concrete outcome your character will achieve when they overcome the obstacle at hand. Usually the character’s goal is pretty obvious in context, but it’s the GM’s job to ask and clarify the goal when necessary.

\begin{qb}``You’re punching him in the face, right? Okay... what do want to get out of this? Do you want to take him out, or just rough him up so he’ll do what you want?''\end{qb}

\subsection{2. The Player Chooses the Action Rating}

The player chooses which \textbf{action rating} to roll, following from what their character is doing on-screen. If you want to roll your \kw{Skirmish} action, then get in a fight. If you want to roll your \kw{Command} action, then order someone around. You can’t roll a given action rating unless your character is presently performing that action in the fiction.

\subsection{3. The GM Sets the Position}

Once the player chooses their action, the GM sets the \textbf{position} for the roll. The position represents how dangerous or troublesome the action might be. There are three positions: \textbf{controlled}, \textbf{risky}, and \textbf{desperate}. To choose a position, the GM looks at the profiles for the positions below and picks one that most closely matches the situation at hand.

\textbf{By default}, an action roll is risky. You wouldn’t be rolling if there was no risk involved. If the situation seems more dangerous, make it desperate. If it seems less dangerous, make it controlled.

\textbf{4. The GM Sets the Effect Level}

The GM assesses the likely \textbf{effect level} of this action, given the factors of the situation. Essentially, the effect level tells us ``how much'' this action can accomplish: will it have \textbf{limited}, \textbf{standard}, or \textbf{great} effect?

\begin{qb}The GM’s choices for effect level and position can be strongly influenced by the player’s choice of action rating. If a player wants to try to make a new friend by \kw{Wrecking} something--0well... maybe that’s possible, but the GM wouldn’t be crazy to say it’s a desperate roll and probably limited effect. Seems like \kw{Consorting} would be a lot better for that. The players are always free to choose the action they perform, but that doesn’t mean all actions should be equally risky or potent.\end{qb}

\textbf{5. Add Bonus Dice}

You can normally get two bonus dice for your action roll (some special abilities might give you additional bonus dice).

For one bonus die, you can get \textbf{assistance} from a teammate. They take 1 stress, say how they help you, and give you +1d.

For another bonus die, you can either \textbf{push yourself} (take 2 stress) or you can accept a \textbf{Devil’s Bargain} (you can’t get dice for both, it’s one or the other).

\subsubsection{The Devil’s Bargain}

PCs in Blades are reckless scoundrels addicted to destructive vices—they don’t always act in their own best interests. To reflect this, the GM or any other player can offer you a bonus die if you accept a Devil’s Bargain. Common Devil’s Bargains include:

\begin{itemize}
\item    Collateral damage, unintended harm.
\item    Sacrifice \kw{Coin} or an item.
\item    Betray a friend or loved one.
\item    Offend or anger a faction.
\item    Start and/or tick a troublesome clock.
\item    Add heat to the crew from evidence or witnesses.
\item    Suffer harm.
\end{itemize}

The Devil’s Bargain occurs regardless of the outcome of the roll. You make the deal, pay the price, and get the bonus die.

The Devil’s Bargain is always a free choice. If you don’t like one, just reject it (or suggest how to alter it so you might consider taking it). You can always just push yourself for that bonus die instead.

If it’s ever needed, the GM has final say over which Devil’s Bargains are valid.

\subsection{6. Roll the Dice and Judge the Result}

Once the goal, action rating, position, and effect have been established, add any bonus dice and roll the dice pool to determine the outcome. (See the sets of possible outcomes, by position, in the table.)

The action roll does a lot of work for you. It tells you how well the character performs as well as how serious the consequences are for them. They might succeed at their action without any consequences (on a \textbf{6}), or they might succeed but suffer consequences (on a \textbf{4/5}), or it might just all go wrong (on a \textbf{1-3}).

On a \textbf{1-3}, it’s up to the GM to decide if the PC’s action has any effect or not, or if it even happens at all. Usually, the action just fails completely, but in some circumstances, it might make sense or be more interesting for the action to have some effect even on a \textbf{1-3} result.

Each \textbf{4/5} and \textbf{1-3} outcome lists suggested \textbf{consequences} for the character. The worse your position, the worse the consequences are. The GM can inflict one or more of these consequences, depending on the circumstances of the action roll. PCs have the ability to avoid or reduce the severity of consequences that they suffer by \textbf{resisting} them.

When you narrate the action after the roll, the GM and player collaborate together to say what happens on-screen. \emph{Tell us how you vault across to the other rooftop. Tell us what you say to the Inspector to convince her. The GM will tell us how she reacts. When you face the Red Sash duelist, what’s your fighting style like? Etc.}

\subsection{Action Roll Summary}

\begin{itemize}
\item A player or GM calls for a roll. Make an \textbf{action roll} when the character performs a dangerous or troublesome action.
\item The player chooses the \textbf{action rating} to roll. Choose the action that matches what the character is doing in the fiction.
\item The GM establishes the \textbf{position} and \textbf{effect level} of the action. The choice of position and effect is influenced strongly by the player’s choice of action.
\item Add up to two bonus dice. 1) \textbf{Assistance} from a teammate. 2) \textbf{Push yourself} (take 2 stress) or accept a \textbf{Devil’s Bargain}.
    Roll the dice pool and judge the outcome. The players and GM narrate the action together. The GM has final say over what happens and inflicts consequences as called for by the position and the result of the roll.
\end{itemize}

{\setlength\fboxsep{1em}
\colorbox{lightgray}{\begin{minipage}{\linewidth}
\begin{LARGE}\textbf{Action Roll}\end{LARGE}

\begin{itemize}
\item    \textbf{1d} for each \textbf{Action} rating dot.
\item    \textbf{+1d} if you have \textbf{Assistance}.
\item    \textbf{+1d} if you \textbf{Push} yourself -or- you accept a \textbf{Devil’s Bargain}.
\end{itemize}

\textbf{Controlled}---You act on your terms. You exploit a dominant advantage.

\begin{itemize}
\item    \textbf{Critical}: You do it with \textbf{increased effect}.
\item    \textbf{6}: You do it.
\item    \textbf{4/5}: You hesitate. Withdraw and try a different approach, or else do it with a minor consequence: a minor \textbf{complication} occurs, you have reduced effect, you suffer lesser harm, you end up in a risky position.
\item    \textbf{1-3}: You falter. Press on by seizing a risky opportunity, or withdraw and try a different approach.
\end{itemize}

\textbf{Risky}---You go head to head. You act under fire. You take a chance.

\begin{itemize}
\item    \textbf{Critical}: You do it with \textbf{increased effect}.
\item    \textbf{6}: You do it.
\item    \textbf{4/5}: You do it, but there’s a consequence: you suffer \textbf{harm}, a \textbf{complication} occurs, you have \textbf{reduced effect}, you end up in a \textbf{desperate} position.
\item    \textbf{1-3}: Things go badly. You suffer \textbf{harm}, a \textbf{complication} occurs, you end up in a \textbf{desperate} position, you \textbf{lose this opportunity}.
\end{itemize}

\textbf{Desperate}---You overreach your capabilities. You’re in serious trouble. 
\begin{itemize}
\item    \textbf{Critical}: You do it with increased effect.
\item    \textbf{6}: You do it.
\item    \textbf{4/5}: You do it, but there’s a consequence: you suffer \textbf{severe harm}, a \textbf{serious complication} occurs, you have \textbf{reduced effect}.
\item    1-3: It’s the worst outcome. You suffer \textbf{severe harm}, a \textbf{serious complication} occurs, you \textbf{lose this opportunity} for action.
\end{itemize}
\vspace{0.5em}
\end{minipage}}}

\subsection{Double-duty Rolls}

Since NPCs don’t roll for their actions, an action roll does double-duty: \textbf{it resolves the action of the PC as well as any NPCs that are involved}. The single roll tells us how those actions interact and which consequences result. On a \textbf{6}, the PC wins and has their effect. On a \textbf{4/5}, it’s a mix—both the PC and the NPC have their effect. On a \textbf{1-3}, the NPC wins and has their effect as a consequence on the PC.


\section{Effect}

In emph{Blades in Penóm}, you achieve goals by taking actions and facing consequences. But how many actions does it take to achieve a particular goal? That depends on the \textbf{effect level} of your actions. The GM judges the effect level using the profiles below. Which one best matches the action at hand---\textbf{great}, \textbf{standard}, or \textbf{limited}? Each effect level indicates the questions that should be answered for that effect, as well as how many segments to tick if you’re using a \textbf{progress clock}.

\begin{center}\begin{tabular}{|l|m{8cm}|c|}
\hline
\multicolumn{2}{|c|}{\textbf{Effects}} & \textbf{Ticks} \\
\hline
\textbf{Great} & \emph{You achieve more than usual. How does the extra effort manifest? What additional benefit do you enjoy?} & 3 \\
\hline
\textbf{Standard} & \emph{You achieve what we’d expect as ``normal'' with this action. Is that enough, or is there more left to do?} & 2\\
\hline
\textbf{Limited} & \emph{You achieve a partial or weak effect. How is your impact diminished? What effort remains to achieve your goal?} & 1 \\
\hline
\end{tabular}\end{center}

\subsection{Assessing Factors}

To assess effect level, first start with your gut feeling, given this situation. Then, if needed, assess three factors that may modify the effect level: \textbf{potency}, \textbf{scale}, and \textbf{quality}. If the PC has an advantage in a given factor, consider a higher effect level. If they have a disadvantage, consider a reduced effect level.
Potency

The potency factor considers particular weaknesses, taking extra time or a bigger risk, or the influence of arcane powers. An infiltrator is more potent if all the lights are extinguished and they move about in the dark.

\subsection{Quality/Tier}

Quality represents the effectiveness of tools, weapons, or other resources, usually summarized by Tier. \textbf{Fine items} count as +1 bonus in quality, stacking with Tier.

\begin{qb}Thorn is picking the lock to a safehouse run of a gang renowned for occult dealings. Her crew is Tier I and she has fine lockpicks—so she’s effectively Tier II. The Occult gang is Tier III. Thorn is outclassed in quality, so her effect will be limited on the lock.\end{qb}

\subsection{Scale}

Scale represents the number of opponents, size of an area covered, scope of influence, etc. Larger scale can be an advantage or disadvantage depending on the situation. In battle, more people are better. When infiltrating, more people are a hindrance.

When considering factors, effect level might be reduced below limited, resulting in zero effect—or increased beyond great, resulting in an \textbf{extreme effect}.

If a PC special ability gives ``+1 effect,'' it comes into play after the GM has assessed the effect level. For example, if you ended up with zero effect, the +1 effect bonus from your \kw{Bodyguard} ability would bump them up to limited effect.

Also, remember that a PC can \textbf{push themselves} (take 2 stress) to get +1 effect on their action.

Every factor won’t always apply to every situation. You don’t have to do an exact accounting every time, either. Use the factors to help you make a stronger judgment call---don’t feel beholden to them.

\subsection{Trading Position for Effect}

After factors are considered and the GM has announced the effect level, a player might want to trade position for effect, or vice versa. For instance, if they’re going to make a risky roll with standard effect (the most common scenario, generally), they might instead want to push their luck and make a desperate roll but with great effect.

This kind of trade-off isn’t included in the effect factors because it’s not an element the GM should assess when setting the effect level. Once the level is set, though, you can always offer the trade-off to the player if it makes sense in the situation.

\begin{qb}``I Prowl across the courtyard and vault over the wall, hiding in the shadows by the canal dock and gondola.''

\vspace{1em}
``I don’t think you can make it across in one quick dash. The scale of the courtyard is a factor here, so your effect will be limited. Let’s say you can get halfway across with this action, then you’ll have to Prowl through the other half of the space (and the rest of the guards there) to reach the other side.''

\vspace{1em}

``I didn’t realize it was that far. Hmmm. Okay, what if I just go as fast as I can. Can I get all the way across if I make a desperate roll?''

\vspace{1em}

``Yep, sounds good to me!''\end{qb}

\subsection{Consequences}

When a PC suffers an effect from an enemy or a dangerous situation, it’s called a \textbf{consequence}. Consequences are the companion to effects. PCs have effect on the world around them and they suffer consequences in return from the risks they face.
