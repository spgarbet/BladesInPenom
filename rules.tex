\chapter{The Rules}

\section{The Core System}

\subsection{Judgment calls}

When you play, you’ll make several key judgment calls. Everyone contributes, but either the players or the GM gets final say for each:

\begin{itemize}
\item Which actions are reasonable as a solution to a problem? Can this person be swayed? Must we get out the tools and tinker with this old rusty lock, or could it also be quietly finessed? The players have final say.
\item How dangerous and how effective is a given action in this circumstance? How risky is this? Can this person be swayed very little or a whole lot? The GM has final say.
\item Which consequences are inflicted to manifest the dangers in a given circumstance? Does this fall from the roof break your leg? Do the constables merely become suspicious or do they already have you trapped? The GM has final say.
\item Does this situation call for a dice roll, and which one? Is your character in position to make an action roll or must they first make a resistance roll to gain initiative? The GM has final say.
\item Which events in the story match the experience triggers for character and clan advancement? Did you express your character’s beliefs, drives, heritage, or background? You tell us. The players have final say.
\end{itemize}

\subsection{Rolling the Dice}

Blades in the Dark uses six-sided dice. You roll several at once and read the \textbf{single highest result}.

\begin{itemize}
\item If the highest die is a \textbf{6}, it’s a \textbf{full success}—things go well. If you roll more than one \textbf{6}, it’s a \textbf{critical success}—you gain some additional advantage.
\item If the highest die is a \textbf{4 or 5}, that’s a \textbf{partial success}—you do what you were trying to do, but there are consequences: trouble, harm, reduced effect, etc.
\item If the highest die is \textbf{1-3}, it’s a \textbf{bad outcome}. Things go poorly. You probably don’t achieve your goal and you suffer complications, too.
\end{itemize}

\begin{qb}
If you ever need to roll but you have zero (or negative) dice, roll two dice and take the single lowest result. You can’t roll a \textbf{critical} when you have zero dice.
\end{qb}

All the dice systems in the game are expressions of this basic format. When you’re first learning the game, you can always ``collapse'' back down to a simple roll to judge how things go. Look up the exact rule later when you have time.

To create a dice pool for a roll, you’ll use a \textbf{trait} (like your \kw{Finesse} or your \kw{Prowess} or your clan’s Tier) and take dice equal to its \textbf{rating}. You’ll usually end up with one to four dice. Even one die is pretty good in this game--a 50\% chance of success. The most common traits you’ll use are the action ratings of the player characters. A player might roll dice for their \kw{Skirmish} action rating when they fight an enemy, for example.

There are four types of rolls that you’ll use most often in the game:

\begin{description}[font=\rmfamily\bfseries\scshape, leftmargin=1cm]
\item[Action roll] When a PC attempts an action that’s dangerous or troublesome, you make an action roll to find out how it goes. Action rolls and their effects and consequences drive most of the game.
\item[Downtime roll] When the PCs are at their leisure after a job, they can perform downtime activities in relative safety. You make downtime rolls to see how much they get done.
\item[Fortune roll] The GM can make a fortune roll to disclaim decision making and leave something up to chance. How loyal is an NPC? How much does the plague spread? How much evidence is burned before the constables kick in the door?
\item[Resistance roll] A player can make a resistance roll when their character suffers a consequence they don’t like. The roll tells us how much stress their character suffers to reduce the severity of a consequence. When you resist that ``Broken Leg'' harm, you take some stress and now it’s only a ``Sprained Ankle'' instead.
\end{description}

\subsection{The Game Structure}

Blades in the Dark has a structure to play, with four parts. By default, the game is in \textbf{free play}—characters talk to each other, they go places, they do things, they make rolls as needed.

When the group is ready, they choose a target for their next operation, then choose a type of plan to employ. This triggers the \emph{engagement roll} (which establishes the situation as the operation starts) and then the game shifts into the \textbf{score} phase.

During the score, the PCs engage the target—they make rolls, overcome obstacles, call for flashbacks, and complete the operation (successfully or not). When the score is finished, the game shifts into the \textbf{downtime} phase.

During the downtime phase, the GM engages the systems for \emph{payoff}, \emph{heat}, and \emph{entanglements}, to determine all the fallout from the score. Then the PCs each get their \emph{downtime activities}, such as indulging their vice to remove stress or working on a long-term project. When all the downtime activities are complete, the game returns to \textbf{free play} and the cycle starts over again.

The phases are a conceptual model to help you organize the game. They’re not meant to be rigid structures that restrict your options (this is why they’re presented as amorphous blobs of ink without hard edges). Think of the phases as a menu of options to fit whatever it is you’re trying to accomplish in play. Each phase suits a different goal.

\section{Actions \& Attributes}

\subsection{Action Ratings}

There are 12 \textbf{actions} in the game that the player characters use to overcome obstacles.

\begin{item3}
\item \kw{Attune}
\item \kw{Command}
\item \kw{Consort}
\item \kw{Finesse}
\item \kw{Hunt}
\item \kw{Prowl}
\item \kw{Skirmish}
\item \kw{Study}
\item \kw{Survey}
\item \kw{Sway}
\item \kw{Tinker}
\item \kw{Wreck}
\end{item3}

Each action has a rating (from 0 to 4) that tells you how many dice to roll when you perform that action. Action ratings don’t just represent skill or training—you’re free to describe how your character performs that action based on the type of person they are. Maybe your character is good at \kw{Command} because they have a scary stillness to them, while another character barks orders and intimidates people with their military bearing.

You choose which action to perform to overcome an obstacle, by describing what your character does. Actions that are poorly suited to the situation may be less effective and may put the character in more danger, but they can still be attempted. Usually, when you perform an action, you’ll make an \textbf{action roll} to see how it turns out.

\subsection{Action Roll}

You make an \textbf{action roll} when your character does something potentially dangerous or troublesome. The possible results of the action roll depend on your character’s \textbf{position}. There are three positions: \textbf{controlled}, \textbf{risky}, and textbf{desperate}. If you’re in a \textbf{controlled} position, the possible consequences are less serious. If you’re in a \textbf{desperate} position, the consequences can be severe. If you’re somewhere in between, it’s \textbf{risky}--usually considered the ``default'' position for most actions.

If there’s no danger or trouble at hand, you don’t make an action roll. You might make a \textbf{fortune} roll or a textbf{downtime} roll or the GM will simply say yes—and you accomplish your goal.

\subsection{Attribute Ratings}

There are three \textbf{attributes} in the game system that the player characters use to resist bad consequences: \kw{Insight}, \kw{Prowess}, and \kw{Resolve}. Each attribute has a rating (from zero to 4) that tells you how many dice to roll when you use that attribute.

The rating for each attribute is equal to the number of dots in the \textbf{first column} under that attribute (see the examples, at right). The more well-rounded your character is with a particular set of actions, the better their attribute rating.

\subsection{Resistance Roll}

Each attribute resists a different type of danger. If you get stabbed, for example, you resist physical harm with your \kw{Prowess} rating. Resistance rolls always succeed—you diminish or deflect the bad result—but the better your roll, the less \textbf{stress} it costs to reduce or avoid the danger.

When the enemy has a big advantage, you’ll need to make a resistance roll before you can take your own action. For example, when you duel the master sword-fighter, she disarms you before you can strike. You need to make a resistance roll to keep hold of your blade if you want to attack her. Or perhaps you face a powerful ghost and attempt to \kw{Attune} with it to control its actions. But before you can make your own roll, you must resist possession from the spirit.

The GM judges the threat level of the enemies and uses these ``preemptive'' resistance rolls as needed to reflect the capabilities of especially dangerous foes.

Find out more about Resistance Rolls in  \S Resistance \& Armor.

FIXME HERE
This character has a Hunt action rating of 1.
Their Insight attribute rating is 1 (the first column of dots). Insight
They also have Prowl 1 and Skirmish 2.
Their Prowess attribute rating is 2. Prowess

\subsection{Actions}

When you \kw{Attune}, you open your mind to arcane power.

\begin{qb}You might communicate with a ghost. You could try to perceive beyond sight in order to better understand your situation (but Surveying might be better).\end{qb}

When you \kw{Command}, you compel swift obedience.

\begin{qb}You might intimidate or threaten to get what you want. You might lead a clan in a group action. You could try to order people around to persuade them (but Consorting might be better).\end{qb}

When you \kw{Consort}, you socialize with friends and contacts.

\begin{qb}You might gain access to resources, information, people, or places. You might make a good impression or win someone over with your charm and style. You might make new friends or connect with your heritage or background. You could try to manipulate your friends with social pressure (but Sway might be better).\end{qb}

When you \kw{Finesse}, you employ dextrous manipulation or subtle misdirection.

\begin{qb}You might pick someone’s pocket. You might handle the controls of a vehicle or direct a mount. You might formally duel an opponent with graceful fighting arts. You could try to employ those arts in a chaotic melee (but Skirmishing might be better). You could try to pick a lock (but Tinkering might be better).\end{qb}

When you \kw{Hunt}, you carefully track a target.

\begin{qb}You might follow a target or discover their location. You might arrange an ambush. You might attack with precision shooting from a distance. You could try to bring your guns to bear in a melee (but Skirmishing might be better).\end{qb}

When you \kw{Prowl}, you traverse skillfully and quietly.

\begin{qb}You might sneak past a guard or hide in the shadows. You might run and leap across the rooftops. You might attack someone from hiding with a back-stab or blackjack. You could try to waylay a victim in the midst of battle (but Skirmishing might be better).\end{qb}

When you \kw{Skirmish}, you entangle a target in close combat so they can’t easily escape.

\begin{qb}You might brawl or wrestle with them. You might hack and slash. You might seize or hold a position in battle. You could try to fight in a formal duel (but Finessing might be better).\end{qb}

When you \kw{Study}, you scrutinize details and interpret evidence.

\begin{qb}You might gather information from documents, newspapers, and books. You might do research on an esoteric topic. You might closely analyze a person to detect lies or true feelings. You could try to examine events to understand a pressing situation (but Surveying might be better).\end{qb}

When you \kw{Survey}, you observe the situation and anticipate outcomes.

\begin{qb}You might spot telltale signs of trouble before it happens. You might uncover opportunities or weaknesses. You might detect a person’s motivations or intentions. You could try to spot a good ambush point (but Hunting might be better).\end{qb}

When you \kw{Sway}, you influence with guile, charm, or argument.

\begin{qb}You might lie convincingly. You might persuade someone to do what you want. You might argue a compelling case that leaves no clear rebuttal. You could try to trick people into affection or obedience (but Consorting or Commanding might be better).\end{qb}

When you \kw{Tinker}, you fiddle with devices and mechanisms.

\begin{qb}You might create a new gadget or alter an existing item. You might pick a lock or crack a safe. You might disable an alarm or trap. You might turn the clockwork devices around the city to your advantage. You could try to use your technical expertise to control a vehicle (but Finessing might be better).\end{qb}

When you \kw{Wreck}, you unleash savage force.

\begin{qb}You might smash down a door or wall with a sledgehammer, or use an explosive to do the same. You might employ chaos or sabotage to create a distraction or overcome an obstacle. You could try to overwhelm an enemy with sheer force in battle (but Skirmishing might be better).\end{qb}

As you can see, many actions overlap with others. This is by design. As a player, you get to choose which action you roll, by saying what your character does. Can you try to \kw{Wreck} someone during a fight? Sure! The GM tells you the position and effect level of your action in this circumstance. As it says, \kw{Skirmish} might be better (less risky or more effective), depending on the situation at hand (sometimes it won’t be better).

\section{Stress \& Trauma}

\subsection{Stress}

Player characters in Blades in the Dark have a special reserve called \textbf{stress}. When they suffer a consequence that they don’t want to accept, they can take stress instead. The result of the \textbf{resistance roll} determines how much stress it costs to avoid a bad outcome.

\begin{qb}During a knife fight, Daniel’s character, Cross, gets stabbed in the chest. Daniel rolls his \kw{Prowess} rating to resist, and gets a \textbf{2}. It costs 6 stress, minus 2 (the result of the resistance roll) to resist the consequences. Daniel marks off 4 stress and describes how Cross survives.\end{qb}

    The GM rules that the harm is reduced by the resistance roll, but not avoided entirely. Cross suffers level 2 harm (``Chest Wound'') instead of level 3 harm (``Punctured Lung'').

\subsection{Pushing Yourself}

You can use stress to push yourself for greater performance. For each bonus you choose below, take \textbf{2 stress} (each can be chosen once for a given action):

\begin{itemize}
\item Add \textbf{+1d} to your roll. (This may be used for an action roll or downtime roll or any other kind of roll where extra effort would help you)
\item Add \textbf{+1 level} to your effect.
\item Take action when you’re incapacitated.
\end{itemize}

\subsection{Trauma}

When a PC marks their last stress box, they suffer a level of \kw{Trauma}. When you take \kw{Trauma}, circle one of your \textbf{trauma conditions} like \emph{Cold, Reckless, Unstable,} etc. They’re all described below.

When you suffer trauma, you’re taken out of action. You’re ``left for dead'' or otherwise dropped out of the current conflict, only to come back later, shaken and drained. When you return, \textbf{you have zero stress} and your vice has been satisfied for the next downtime.

\textbf{Trauma conditions are permanent}. Your character acquires the new personality quirk indicated by the condition, and can earn \kw{xp} by using it to cause trouble. \textbf{When you mark your fourth trauma condition}, your character cannot continue as a daring scoundrel. You must retire them to a different life or send them to prison to take the fall for the clan’s \kw{Wanted Level}.

\subsection{Trauma Conditions}

\begin{description}
\item[Cold] You’re not moved by emotional appeals or social bonds.
\item[Haunted] You’re often lost in reverie, reliving past horrors, seeing things.
\item[Obsessed] You’re enthralled by one thing: an activity, a person, an ideology.
\item[Paranoid] You imagine danger everywhere; you can’t trust others.
\item[Reckless] You have little regard for your own safety or best interests.
\item[Soft] You lose your edge; you become sentimental, passive, gentle.
\item[Unstable] Your emotional state is volatile. You can instantly rage, or fall into despair, act impulsively, or freeze up.
\item[Vicious] You seek out opportunities to hurt people, even for no good reason.
\end{description}

\section{Progress Clocks}

\begin{qb}Sneaking into the constables watch tower? Make a clock to track the alert level of the patrolling guards. When the PCs suffer consequences from partial successes or missed rolls, fill in segments on the clock until the alarm is raised.\end{qb}

\begin{wrapfigure}[15]{r}{0.2\textwidth}\begin{center}\includegraphics[scale=0.2]{img/progress-clocks.png}\end{center}\end{wrapfigure}

A \textbf{progress clock} is a circle divided into segments (see examples at right). Draw a progress clock when you need to track ongoing effort against an obstacle or the approach of impending trouble.


Generally, the more complex the problem, the more segments in the progress clock.

A complex obstacle is a 4-segment clock. A more complicated obstacle is a 6-clock. A daunting obstacle is an 8-segment clock.

When you create a clock, make it about the \textbf{obstacle}, not the method. The clocks for an infiltration should be ``Interior Patrols'' and ``The Tower,'' not ``Sneak Past the Guards'' or ``Climb the Tower.'' The patrols and the tower are the obstacles­—the PCs can attempt to overcome them in a variety of ways.

Complex enemy threats can be broken into several ``layers,'' each with its own progress clock. For example, the dockside clan’ HQ might have a ``Perimeter Security'' clock, an ``Interior Guards'' clock, and an ``Office Security'' clock. The clan would have to make their way through all three layers to reach the clan leaders personal safe and valuables within.

Remember that a clock tracks progress. It reflects the fictional situation, so the group can gauge how they’re doing. A clock is like a speedometer in a car. It shows the speed of the vehicle—it doesn’t determine the speed.

\subsection{Simple Obstacles}

Not every situation and obstacle requires a clock. Use clocks when a situation is complex or layered and you need to track something over time—otherwise, resolve the result of an action with a single roll.

\subsubsection{Danger Clocks}

The GM can use a clock to represent a progressive danger, like suspicion growing during a seduction, the proximity of pursuers in a chase, or the alert level of guards on patrol. In this case, when a complication occurs, the GM ticks one, two, or three segments on the clock, depending on the consequence level. When the clock is full, the danger comes to fruition—the guards hunt down the intruders, activate an alarm, release the hounds, etc.

\subsubsection{Racing Clocks}

Create two opposed clocks to represent a race. The PCs might have a progress clock called ``Escape'' while the constables have a clock called ``Cornered.'' If the PCs finish their clock before the constables fill theirs, they get away. Otherwise, they’re cornered and can’t flee. If both complete at the same time, the PCs escape to their lair, but the hunting officers are outside!

You can also use racing clocks for an environmental hazard. Maybe the PCs are trying to complete the ``Search'' clock to find the lockbox on the sinking ship before the GM fills the ``Sunk'' clock and the vessel goes down.

\subsubsection{Linked Clocks}

You can make a clock that unlocks another clock once it’s filled. For example, the GM might make a linked clock called ``Trapped'' after an ``Alert'' clock fills up. When you fight a veteran warrior, she might have a clock for her ``Defense'' and then a linked clock for ``Vulnerable.'' Once you overcome the ``Defense'' clock, then you can attempt to overcome the ``Vulnerable'' clock and defeat her. You might affect the ``Defense'' clock with violence in a knife-fight, or you lower her defense with deception if you have the opportunity. As always, the method of action is up to the players and the details of the fiction at hand.

\subsubsection{Mission Clocks}

The GM can make a clock for a time-sensitive mission, to represent the window of opportunity you have to complete it. If the countdown runs out, the mission is scrubbed or changes—the target escapes, the household wakes up for the day, etc.

\subsubsection{Tug-of-war Clocks}

You can make a clock that can be filled and emptied by events, to represent a back-and-forth situation. You might make a ``Revolution!'' clock that indicates when the refugees start to riot over poor treatment. Some events will tick the clock up and some will tick it down. Once it fills, the revolution begins. A tug-of-war clock is also perfect for an ongoing turf war between two clan or factions.

\subsection{Long-term Project}

Some projects will take a long time. A basic long-term project (like tinkering up a new feature for a device) is eight segments. Truly long-term projects (like creating a new designer drug) can be two, three, or even four clocks, representing all the phases of development, testing, and final completion. Add or subtract clocks depending on the details of the situation and complexity of the project.

A long-term project is a good catch-all for dealing with any unusual player goal, including things that circumvent or change elements of the mechanics or the setting.

\subsection{Faction Clocks}

Each faction has a long-term goal. When the PCs have \textbf{downtime}, the GM ticks forward the faction clocks that they’re interested in. In this way, the world around the PCs is dynamic and things happen that they’re not directly connected to, changing the overall situation in the city and creating new opportunities and challenges.

The PCs may also directly affect NPC faction clocks, based on the missions and scores they pull off. Discuss known faction projects that they might aid or interfere with, and also consider how a PC operation might affect the NPC clocks, whether the players intended it or not.

\section{Action Roll}


When a player character does something challenging, we make an \textbf{action roll} to see how it turns out. An action is challenging if there’s an obstacle to the PC’s goal that’s dangerous or troublesome in some way. We don’t make an action roll unless the PC is put to the test. If their action is something that we’d expect them to simply accomplish, then we don’t make an action roll.

\begin{qb}Each game group will have their own ideas about what ``challenging'' means. This is good! It’s something that establishes the tone and style of your Blades series.\end{qb}

To make an action roll, we go through six steps. In play, they flow together somewhat, but let’s break each one down here for clarity.

\begin{enumerate}
\item    The player states their \textbf{goal} for the action.
\item    The player chooses the \textbf{action rating}.
\item    The GM sets the \textbf{position} for the roll.
\item    The GM sets the \textbf{effect level} for the action.
\item    Add \textbf{bonus dice}.
\item    The player rolls the dice and we judge the result.
\end{enumerate}

\subsection{1. The Player States Their Goal}

Your goal is the concrete outcome your character will achieve when they overcome the obstacle at hand. Usually the character’s goal is pretty obvious in context, but it’s the GM’s job to ask and clarify the goal when necessary.

\begin{qb}``You’re punching him in the face, right? Okay... what do want to get out of this? Do you want to take him out, or just rough him up so he’ll do what you want?''\end{qb}

\subsection{2. The Player Chooses the Action Rating}

The player chooses which \textbf{action rating} to roll, following from what their character is doing on-screen. If you want to roll your \kw{Skirmish} action, then get in a fight. If you want to roll your \kw{Command} action, then order someone around. You can’t roll a given action rating unless your character is presently performing that action in the fiction.

\subsection{3. The GM Sets the Position}

Once the player chooses their action, the GM sets the \textbf{position} for the roll. The position represents how dangerous or troublesome the action might be. There are three positions: \textbf{controlled}, \textbf{risky}, and \textbf{desperate}. To choose a position, the GM looks at the profiles for the positions below and picks one that most closely matches the situation at hand.

\textbf{By default}, an action roll is risky. You wouldn’t be rolling if there was no risk involved. If the situation seems more dangerous, make it desperate. If it seems less dangerous, make it controlled.

\textbf{4. The GM Sets the Effect Level}

The GM assesses the likely \textbf{effect level} of this action, given the factors of the situation. Essentially, the effect level tells us ``how much'' this action can accomplish: will it have \textbf{limited}, \textbf{standard}, or \textbf{great} effect?

\begin{qb}The GM’s choices for effect level and position can be strongly influenced by the player’s choice of action rating. If a player wants to try to make a new friend by \kw{Wrecking} something--0well... maybe that’s possible, but the GM wouldn’t be crazy to say it’s a desperate roll and probably limited effect. Seems like \kw{Consorting} would be a lot better for that. The players are always free to choose the action they perform, but that doesn’t mean all actions should be equally risky or potent.\end{qb}

\textbf{5. Add Bonus Dice}

You can normally get two bonus dice for your action roll (some special abilities might give you additional bonus dice).

For one bonus die, you can get \textbf{assistance} from a teammate. They take 1 stress, say how they help you, and give you +1d.

For another bonus die, you can either \textbf{push yourself} (take 2 stress) or you can accept a \textbf{Devil’s Bargain} (you can’t get dice for both, it’s one or the other).

\subsubsection{The Devil’s Bargain}

PCs in Blades are reckless scoundrels addicted to destructive vices—they don’t always act in their own best interests. To reflect this, the GM or any other player can offer you a bonus die if you accept a Devil’s Bargain. Common Devil’s Bargains include:

\begin{itemize}
\item    Collateral damage, unintended harm.
\item    Sacrifice \kw{Coin} or an item.
\item    Betray a friend or loved one.
\item    Offend or anger a faction.
\item    Start and/or tick a troublesome clock.
\item    Add heat/dishonor to the clan from evidence or witnesses.
\item    Suffer harm.
\end{itemize}

The Devil’s Bargain occurs regardless of the outcome of the roll. You make the deal, pay the price, and get the bonus die.

The Devil’s Bargain is always a free choice. If you don’t like one, just reject it (or suggest how to alter it so you might consider taking it). You can always just push yourself for that bonus die instead.

If it’s ever needed, the GM has final say over which Devil’s Bargains are valid.

\subsection{6. Roll the Dice and Judge the Result}

Once the goal, action rating, position, and effect have been established, add any bonus dice and roll the dice pool to determine the outcome. (See the sets of possible outcomes, by position, in the table.)

The action roll does a lot of work for you. It tells you how well the character performs as well as how serious the consequences are for them. They might succeed at their action without any consequences (on a \textbf{6}), or they might succeed but suffer consequences (on a \textbf{4/5}), or it might just all go wrong (on a \textbf{1-3}).

On a \textbf{1-3}, it’s up to the GM to decide if the PC’s action has any effect or not, or if it even happens at all. Usually, the action just fails completely, but in some circumstances, it might make sense or be more interesting for the action to have some effect even on a \textbf{1-3} result.

Each \textbf{4/5} and \textbf{1-3} outcome lists suggested \textbf{consequences} for the character. The worse your position, the worse the consequences are. The GM can inflict one or more of these consequences, depending on the circumstances of the action roll. PCs have the ability to avoid or reduce the severity of consequences that they suffer by \textbf{resisting} them.

When you narrate the action after the roll, the GM and player collaborate together to say what happens on-screen. \emph{Tell us how you vault across to the other rooftop. Tell us what you say to the Inspector to convince her. The GM will tell us how she reacts. When you face the Red Sash duelist, what’s your fighting style like? Etc.}

\subsection{Action Roll Summary}

\begin{itemize}
\item A player or GM calls for a roll. Make an \textbf{action roll} when the character performs a dangerous or troublesome action.
\item The player chooses the \textbf{action rating} to roll. Choose the action that matches what the character is doing in the fiction.
\item The GM establishes the \textbf{position} and \textbf{effect level} of the action. The choice of position and effect is influenced strongly by the player’s choice of action.
\item Add up to two bonus dice. 1) \textbf{Assistance} from a teammate. 2) \textbf{Push yourself} (take 2 stress) or accept a \textbf{Devil’s Bargain}.
    Roll the dice pool and judge the outcome. The players and GM narrate the action together. The GM has final say over what happens and inflicts consequences as called for by the position and the result of the roll.
\end{itemize}

\gbox{Action Roll}{
\begin{itemize}
\item    \textbf{1d} for each \textbf{Action} rating dot.
\item    \textbf{+1d} if you have \textbf{Assistance}.
\item    \textbf{+1d} if you \textbf{Push} yourself -or- you accept a \textbf{Devil’s Bargain}.
\end{itemize}

\textbf{Controlled}---You act on your terms. You exploit a dominant advantage.

\begin{itemize}
\item    \textbf{Critical}: You do it with \textbf{increased effect}.
\item    \textbf{6}: You do it.
\item    \textbf{4/5}: You hesitate. Withdraw and try a different approach, or else do it with a minor consequence: a minor \textbf{complication} occurs, you have reduced effect, you suffer lesser harm, you end up in a risky position.
\item    \textbf{1-3}: You falter. Press on by seizing a risky opportunity, or withdraw and try a different approach.
\end{itemize}

\textbf{Risky}---You go head to head. You act under fire. You take a chance.

\begin{itemize}
\item    \textbf{Critical}: You do it with \textbf{increased effect}.
\item    \textbf{6}: You do it.
\item    \textbf{4/5}: You do it, but there’s a consequence: you suffer \textbf{harm}, a \textbf{complication} occurs, you have \textbf{reduced effect}, you end up in a \textbf{desperate} position.
\item    \textbf{1-3}: Things go badly. You suffer \textbf{harm}, a \textbf{complication} occurs, you end up in a \textbf{desperate} position, you \textbf{lose this opportunity}.
\end{itemize}

\textbf{Desperate}---You overreach your capabilities. You’re in serious trouble. 
\begin{itemize}
\item    \textbf{Critical}: You do it with increased effect.
\item    \textbf{6}: You do it.
\item    \textbf{4/5}: You do it, but there’s a consequence: you suffer \textbf{severe harm}, a \textbf{serious complication} occurs, you have \textbf{reduced effect}.
\item    1-3: It’s the worst outcome. You suffer \textbf{severe harm}, a \textbf{serious complication} occurs, you \textbf{lose this opportunity} for action.
\end{itemize}
}

\subsection{Double-duty Rolls}

Since NPCs don’t roll for their actions, an action roll does double-duty: \textbf{it resolves the action of the PC as well as any NPCs that are involved}. The single roll tells us how those actions interact and which consequences result. On a \textbf{6}, the PC wins and has their effect. On a \textbf{4/5}, it’s a mix—both the PC and the NPC have their effect. On a \textbf{1-3}, the NPC wins and has their effect as a consequence on the PC.


\section{Effect}

In emph{Blades in Penóm}, you achieve goals by taking actions and facing consequences. But how many actions does it take to achieve a particular goal? That depends on the \textbf{effect level} of your actions. The GM judges the effect level using the profiles below. Which one best matches the action at hand---\textbf{great}, \textbf{standard}, or \textbf{limited}? Each effect level indicates the questions that should be answered for that effect, as well as how many segments to tick if you’re using a \textbf{progress clock}.

\begin{center}\begin{tabular}{|l|m{8cm}|c|}
\hline
\multicolumn{2}{|c|}{\textbf{Effects}} & \textbf{Ticks} \\
\hline
\textbf{Great} & \emph{You achieve more than usual. How does the extra effort manifest? What additional benefit do you enjoy?} & 3 \\
\hline
\textbf{Standard} & \emph{You achieve what we’d expect as ``normal'' with this action. Is that enough, or is there more left to do?} & 2\\
\hline
\textbf{Limited} & \emph{You achieve a partial or weak effect. How is your impact diminished? What effort remains to achieve your goal?} & 1 \\
\hline
\end{tabular}\end{center}

\subsection{Assessing Factors}

To assess effect level, first start with your gut feeling, given this situation. Then, if needed, assess three factors that may modify the effect level: \textbf{potency}, \textbf{scale}, and \textbf{quality}. If the PC has an advantage in a given factor, consider a higher effect level. If they have a disadvantage, consider a reduced effect level.
Potency

The potency factor considers particular weaknesses, taking extra time or a bigger risk, or the influence of arcane powers. An infiltrator is more potent if all the lights are extinguished and they move about in the dark.

\subsection{Quality/Tier}

Quality represents the effectiveness of tools, weapons, or other resources, usually summarized by Tier. \textbf{Fine items} count as +1 bonus in quality, stacking with Tier.

\begin{qb}Thorn is picking the lock to a safehouse run by a clan renowned for occult dealings. Her clan is Tier I and she has fine lockpicks—so she’s effectively Tier II. The Occult clan is Tier III. Thorn is outclassed in quality, so her effect will be limited on the lock.\end{qb}

\subsection{Scale}

Scale represents the number of opponents, size of an area covered, scope of influence, etc. Larger scale can be an advantage or disadvantage depending on the situation. In battle, more people are better. When infiltrating, more people are a hindrance.

When considering factors, effect level might be reduced below limited, resulting in zero effect—or increased beyond great, resulting in an \textbf{extreme effect}.

If a PC special ability gives ``+1 effect,'' it comes into play after the GM has assessed the effect level. For example, if you ended up with zero effect, the +1 effect bonus from your \kw{Bodyguard} ability would bump them up to limited effect.

Also, remember that a PC can \textbf{push themselves} (take 2 stress) to get +1 effect on their action.

Every factor won’t always apply to every situation. You don’t have to do an exact accounting every time, either. Use the factors to help you make a stronger judgment call---don’t feel beholden to them.

\subsection{Trading Position for Effect}

After factors are considered and the GM has announced the effect level, a player might want to trade position for effect, or vice versa. For instance, if they’re going to make a risky roll with standard effect (the most common scenario, generally), they might instead want to push their luck and make a desperate roll but with great effect.

This kind of trade-off isn’t included in the effect factors because it’s not an element the GM should assess when setting the effect level. Once the level is set, though, you can always offer the trade-off to the player if it makes sense in the situation.

\begin{qb}``I Prowl across the courtyard and vault over the wall, hiding in the shadows by the canal dock and gondola.''

``I don’t think you can make it across in one quick dash. The scale of the courtyard is a factor here, so your effect will be limited. Let’s say you can get halfway across with this action, then you’ll have to Prowl through the other half of the space (and the rest of the guards there) to reach the other side.''


``I didn’t realize it was that far. Hmmm. Okay, what if I just go as fast as I can. Can I get all the way across if I make a desperate roll?''


``Yep, sounds good to me!''\end{qb}

\subsection{Consequences}

When a PC suffers an effect from an enemy or a dangerous situation, it’s called a \textbf{consequence}. Consequences are the companion to effects. PCs have effect on the world around them and they suffer consequences in return from the risks they face.

\section{Setting Position \& Effect}


The GM sets position and effect for an action roll at the same time, after the player says what they’re doing and chooses their action. Usually, \textbf{Risky / Standard} is the default combination, modified by the action being used, the strength of the opposition, and the effect factors.

The ability to set position and effect as independent variables gives you nine combinations to choose from, to help you convey a wide array of fictional circumstances.

\emph{For example, if a character is facing off alone against a small enemy clan, the situation might be:}

\begin{itemize}
\item \emph{She fights the clan straight up, rushing into their midst, hacking away in a wild} \kw{Skirmish}. In this case, being threatened by the larger force lowers her position to indicate greater risk, and the scale of the clan reduces her effect (Desperate / Limited).
\item \emph{She fights the clan from a choke-point, like a narrow alleyway where their numbers can’t overwhelm her at once.} She’s not threatened by several at once, so her risk is similar to a one-on-one fight, but there’s still a lot of enemies to deal with, so her effect is reduced (Risky / Limited).
\item \emph{She doesn’t fight the clan, instead trying to maneuver her way past them and escape.} She’s still under threat from many enemy attacks, so her position is worse, but if the ground is open and the clan can’t easily corral her, then her effect for escaping isn’t reduced (Desperate / Standard). If she had some immediate means of escape (like leaping onto a speeding carriage), then her effect might even be increased (Desperate / Great).
\item \emph{The clan isn’t aware of her yet—she’s set up in a sniper position on a nearby roof. She takes a shot against one of them.} Their greater numbers aren’t a factor, so her effect isn’t reduced, and she’s not immediately in any danger (Controlled / Great). Maybe instead she wants to fire off a salvo of suppressing fire against the whole clan, in which case their scale applies (Controlled / Limited). If the clan is on guard for potential trouble, her position is more dangerous (Risky / Great). If the clan is alerted to a sniper, then the effect may be reduced further, as they scatter and take cover (Risky / Limited). If the clan is able to muster covering fire while they fall back to a safe position, then things are even worse for our scoundrel (Desperate / Limited).
\end{itemize}

\section{Consequences \& Harm}


Enemy actions, bad circumstances, or the outcome of a roll can inflict \textbf{consequences} on a PC. There are five types (at right).

A given circumstance might result in one or more consequences, depending on the situation. The GM determines the consequences, following from the fiction and the style and tone established by the game group.

\subsection{Reduced Effect}

This consequence represents impaired performance. The PC’s action isn’t as effective as they’d anticipated. You hit him, but it’s only a flesh wound. She accepts the forged invitation, but she’ll keep her eye on you throughout the night. You’re able to scale the wall, but it’s slow going—you’re only halfway up. This consequence essentially reduces the effect level of the PC’s action by one after all other factors are accounted for.

\subsection{Complication}

This consequence represents trouble, mounting danger, or a new threat. The GM might introduce an immediate problem that results from the action right now: the room catches fire, you’re disarmed, the clan takes +1 \kw{Heat} from evidence or witnesses, you lose status with a faction, the target evades you and now it’s a chase, reinforcements arrive, etc.

Or the GM might tick a clock for the complication, instead. Maybe there’s a clock for the alert level of the guards at the manor. Or maybe the GM creates a new clock for the suspicion of the noble guests at the masquerade party and ticks it. Fill one tick on a clock for a minor complication or two ticks for a standard complication.

A \textbf{serious complication} is more severe: reinforcements surround and trap you, the room catches fire and falling ceiling beams block the door, your weapon is broken, the clan suffers +2 \kw{Heat}, your target escapes out of sight, etc. Fill three ticks on a clock for a serious complication.

\textbf{Don’t inflict a complication that negates a successful roll}. If a PC tries to corner an enemy and gets a \textbf{4/5}, don’t say that the enemy escapes. The player’s roll succeeded, so the enemy is cornered... maybe the PC has to wrestle them into position and during the scuffle the enemy grabs their gun.

\subsection{Lost Opportunity}

This consequence represents shifting circumstance. You had an opportunity to achieve your goal with this action, but it slips away. To try again, you need a new approach—usually a new form of action or a change in circumstances. Maybe you tried to \kw{Skirmish} with the noble to trap her on the balcony, but she evades your maneuver and leaps out of reach. If you want to trap her now you’ll have to try another way—maybe by \kw{Swaying} her with your roguish charm.

\subsection{Worse Position}

This consequence represents losing control of the situation—the action carries you into a more dangerous position. Perhaps you make the leap across to the next rooftop, only to end up dangling by your fingertips. You haven’t failed, but you haven’t succeeded yet, either. You can try again, re-rolling at the new, worse position. This is a good consequence to choose to show escalating action. A situation might go from controlled, to risky, to desperate as the action plays out and the PC gets deeper and deeper in trouble.

\subsection{Harm}

This consequence represents a long-lasting debility (or death). When you suffer harm, record the specific injury on your character sheet equal to the level of harm you suffer. If you suffer \textbf{lesser harm}, record it in the bottom row. If you suffer \textbf{moderate harm}, write it in the middle row. If you suffer \textbf{severe harm}, record it in the top row. See examples of harm and the harm tracker, below.

Your character suffers the penalty indicated at the end of the row if any or all harm recorded in that row applies to the situation at hand. So, if you have ``Drained'' and ``Battered'' harm in the bottom row, you’ll suffer reduced effect when you try to run away from the constables. When you’re impaired by harm in the top row (severe harm, level 3), your character is incapacitated and can’t do anything unless you have help from someone else or push yourself to perform the action.

If you need to mark a harm level, but the row is already filled, the harm moves up to the next row above. So, if you suffered standard harm (level 2) but had no empty spaces in the second row, you’d have to record severe harm (level 3), instead. If you run out of spaces on the top row and need to mark harm there, your character suffers a \textbf{catastrophic}, \textbf{permanent consequence} (loss of a limb, sudden death, etc., depending on the circumstances).

\begin{figure}[H]
\centering\begin{tabular}{|p{0.7cm}|p{3cm}|p{3cm}|m{1.5cm}|}
\hline
\multicolumn{4}{|l|}{\small HARM} \\
\hline
3 & \multicolumn{2}{|l|}{\emph{\Large Shattered Right Leg}} & {\small NEED HELP} \\
\hline
\rule{0pt}{13pt}\rule[-1em]{0pt}{9pt}2 & & & {\centering\small -1D} \\ 
\hline
1 & \emph{\centering\Large Drained} & \emph{\Large Battered} & {\centering\small REDUCED EFFECT} \\
\hline
\end{tabular}
\caption[Harm Table]%
  {This character has three harm: a ``Shattered Right Leg'' (level 3) plus ``Drained'' and ``Battered'' (level 1). If they suffer another level 1 harm, it will move up to level 2. If they suffer another level 3 harm, it will move up to level 4: Fatal.}
\label{harm_fig}
\end{figure}


\subsection{Harm examples}

\begin{description}
\item[Fatal (4)] Electrocuted, Drowned, Stabbed in the Heart.
\item[Severe (3)] Impaled, Broken Leg, Shot in Chest, Badly Burned, Terrified.
\item[Moderate (2)] Exhausted, Deep Cut to Arm, Concussion, Panicked, Seduced.
\item[Lesser (1)] Battered, Drained, Distracted, Scared, Confused.
\end{description}

Harm like ``Drained'' or ``Exhausted'' can be a good fallback consequence if there’s nothing else threatening a PC (like when they spend all night Studying those old books, looking for any clues to their enemy’s weaknesses before he strikes).

\section{Resistance \& Armor}

When your PC suffers a consequence that you don’t like, you can choose to resist it. Just tell the GM, “No, I don’t think so. I’m resisting that.” Resistance is always automatically effective—the GM will tell you if the consequence is reduced in severity or if you avoid it entirely. Then, you’ll make a \textbf{resistance roll} to see how much stress your character suffers as a result of their resistance.

You make the roll using one of your character’s \textbf{attributes} (\kw{Insight}, \kw{Prowess}, or \kw{Resolve}). The GM chooses the attribute, based on the nature of consequences:

\begin{description}
\item[Insight] Consequences from deception or understanding.
\item[Prowess] Consequences from physical strain or injury.
\item[Resolve] Consequences from mental strain or willpower.
\end{description}

Your character suffers \textbf{6} stress when they resist, \textbf{minus the highest die result from the resistance roll}. So, if you rolled a \textbf{4}, you’d suffer 2 stress. If you rolled a \textbf{6}, you’d suffer zero stress. If you get a \kw{Critical} result, you also \textbf{clear 1 stress}.

\begin{qb}Ian’s character, Silas, is in a desperate \kw{Skirmish} with several duelists and one of them lands a blow with their sword. Since the position was desperate, the GM inflicts severe harm (modified by any other factors). They tell Ian to record level 3 harm, ``Gut Stabbed'' on Silas’s sheet. Ian decides to resist the harm, instead. The GM says he can reduce the harm by one level if he resists it. Ian rolls 3d for Silas’s \kw{Prowess} attribute and gets a \textbf{5}. Silas takes 1 stress and the harm is reduced to level 2, “Cut to the Ribs.”\end{qb}

Usually, a resistance roll reduces the \textbf{severity of a consequence}. If you’re going to suffer fatal harm, for example, a resistance roll would reduce the harm to severe, instead. Or if you got a complication when you were sneaking into the manor house, and the GM was going to mark three ticks on the “Alert” clock, she’d only mark two (or maybe one) if you resisted the complication.

\textbf{You may only roll against a given consequence once.}

The GM also has the option to rule that your character \textbf{completely avoids} the consequence. For instance, maybe you’re in a sword fight and the consequence is getting disarmed. When you resist, the GM says that you avoid that consequence completely: you keep hold of your weapon.

\textbf{By adjusting which consequences are reduced vs. which are avoided, the GM establishes the overall tone of your game}. For a more daring game, most consequences will be avoided. For a grittier game, most consequences will only be reduced with resistance.

The GM may also threaten several consequences at once, then the player may choose which ones to resist (and make rolls for each).

\begin{qb}
``She stabs you and then leaps off the balcony. Level 2 harm and you lose the opportunity to catch her with fighting.''

``I’ll resist losing the opportunity by grappling her as she attacks. She can stab me, but I don’t want to let her escape.''
\end{qb}

Once you decide to resist a consequence and roll, you suffer the stress indicated. You can’t roll first and see how much stress you’ll take, then decide whether or not to resist.

\gbox{Resistance Roll}{
\textbf{1d} for each \textbf{Attribute} rating.
\vspace{0.5em}

You \textbf{reduce} or \textbf{avoid} the effects of the consequence (GM chooses).
\vspace{0.5em}

\textbf{Suffer 6 stress minus the highest die result}.
\vspace{0.5em}

\kw{Critical}: Clear 1 stress.
}

\subsection{Armor}

If you have a type of \textbf{armor} that applies to the situation, you can mark an armor box to reduce or avoid a consequence, instead of rolling to resist.

\begin{qb}Silas is taking level 2 harm, ``Cut to the Ribs,'' and the fight isn’t even over yet, so Ian decides to use Silas’s armor to reduce the harm. He marks the armor box and the harm becomes level 1, ``Bruised.'' If Silas was wearing heavy armor, he could mark a second armor box and reduce the harm again, to zero.\end{qb}

When an armor box is marked, it can’t be used again until it’s restored. All of your armor is restored when you choose your \textbf{load} for the next score.

\subsection{Death}

There are a couple ways for a PC to die:
\begin{itemize}
\item If they suffer level 4 fatal harm and they don’t resist it, they die. \emph{Sometimes this is a choice a player wants to make, because they feel like it wouldn’t make sense for the character to survive or it seems right for their character to die here.}
\item If they need to record harm at level 3 and it’s already filled, they suffer a catastrophic consequence, which might mean sudden death (depending on the circumstances).
\end{itemize}

When your character dies, you have options:
\begin{itemize}
\item You can create a new character to play.
\item Maybe you ``promote'' one of the NPC clan members to a PC, or create a brand new character who joins the clan.
\end{itemize}

\section{Fortune Roll}

The fortune roll is a tool the GM can use to disclaim decision making. You use a fortune roll in two different ways:

\textbf{When you need to make a determination about a situation the PCs aren’t directly involved in} and don’t want to simply decide the outcome.

\begin{qb}Two rival gangs are fighting. How does that turn out? The GM makes a fortune roll for each of them. One gets a good result but the other gets limited effect. The GM decides that the first gang takes over some of their rivals’ turf but suffer some injuries during the skirmish.\end{qb}

\textbf{When an outcome is uncertain}, but no other roll applies to the situation at hand.

\begin{qb}While pilfering the workshop of an alchemist, Cross is possessed by a vengeful ghost. As control of his body slips away, Cross grabs a random potion bottle and drinks it down. Will the arcane concoction have an effect on the spirit? Will it poison Nock to death? Who knows? The GM makes a fortune roll to see how it turns out.\end{qb}

When you make a fortune roll you may assess \textbf{any trait rating} to determine the dice pool of the roll.

\begin{itemize}
\item When a faction takes an action with uncertain outcome, you might use their \kw{Tier} rating to make a fortune roll.
\item When a clan operates independently, use their \textbf{quality} rating for a fortune roll.
\item When a supernatural power manifests with uncertain results, you might use its \textbf{magnitude} for a fortune roll.
\item When a PC \textbf{gathers information}, you might make a fortune roll using their \textbf{action rating} to determine the amount of the info they get.
\end{itemize}

If no trait applies, roll 1d for sheer luck or create a dice pool (from one to four) based on the situation at hand. If two parties are directly opposed, make a fortune roll for each side to see how they do, then assess the outcome of the situation by comparing their performance levels.

The fortune roll is also a good tool to help the GM manage all the various moving parts of the world. Sometimes a quick roll is enough to answer a question or inspire an idea for what might happen next.

Other examples of fortune rolls:

\begin{itemize}
\item    The PCs instigate a war between two factions, then sit back and watch the fireworks. How does it turn out? Does either side dominate? Are they both made vulnerable by the conflict? Make a few fortune rolls to find out.
\item    A strange sickness is sweeping the city. How badly is a crime ridden district hit by the outbreak? The GM assigns a magnitude to the arcane plague, and makes a fortune roll to judge the extent of its contamination.
\item    The Hound stakes out a good spot and makes a sniper shot against a gang leader when he enters his office. The controlled \kw{Hunt} roll is a success, but is great effect enough to instantly kill a grizzled gang leader? Instead of making a progress clock for his mortality, the GM decides to use a simple fortune roll with his ``toughness'' as a trait to see if he can possibly survive the attack. The roll is a \textbf{4/5}: the bullet misses his heart, but hits him in the lung—it’s a mortal wound. He’s on death’s door, with only hours to live, unless his gang can get an expert physicker to him in time.
\item    Inspectors are putting a case together against the PC clan. How quickly will their evidence result in arrests? The clan’s \kw{Wanted Level} counts as a major advantage for the inspectors.
\item    The PCs face off in a skirmish with a veteran demon hunter captain and her clan. The tide of battle goes in the PCs’ favor, and many clan members are killed. One of the players asks if the captain will surrender to spare the rest of her clan’s lives. The GM isn’t sure. How cold-hearted is this veteran hunter? She’s stared giant demons in the eye without flinching... is there anything human left inside her? The GM makes a 2d fortune roll for ``human feelings'' to see if a spark of compassion remains in heart. If so, maybe one of the PCs can roll to \kw{Consort}, \kw{Sway}, or \kw{Command} her to stand down.
\end{itemize}

\gbox{Fortune Roll}{
\begin{itemize}
\item    \textbf{1d} for each \textbf{Trait} rating.
\item    \textbf{+1d} for each \textbf{Major} advantage.
\item    \textbf{-1d} for each \textbf{Major} disadvantage.
\end{itemize}
\hrule
\begin{itemize}
\item    \textbf{Critical}: Exceptional result / Great, extreme effect.
\item    \textbf{6}: Good result / Standard,  full effect.
\item    \textbf{4/5}: Mixed result / Limited, partial  effect.
\item    \textbf{1-3}: Bad result / Poor, little effect.
\end{itemize}
}

\section{Gathering Information}


The flow of information from the GM to the players about the fictional world is very important in a roleplaying game. By default, the GM tells the players what their characters perceive, suspect, and intuit. But there’s just too much going on to say \emph{everything}---it would take forever and be boring, too. The players have a tool at their disposal to more fully investigate the fictional world.

When you want to know something specific about the fictional world, your character can \textbf{gather} information. The GM will ask you \textbf{how} your character gathers the info (or how they learned it in the past).

If it’s common knowledge, the GM will simply answer your questions. If there’s an obstacle to the discovery of the answer, an action roll is called for. If it’s not common knowledge but there’s no obstacle, a simple fortune roll determines the quality of the information you gather.

Each attempt to gather information takes time. If the situation allows, you can try again if you don’t initially get all the info that you want. But often, the opportunity is fleeting, and you’ll only get one chance to roll for that particular question.

Some example questions are on the bottom of the character sheet. The GM always answers honestly, but with a level of detail according to the level of effect.

The most common gather information actions are \kw{Surveying} the situation to reveal or anticipate what’s going on and \kw{Studying} a person to understand what they intend to do or what they’re really thinking.

Sometimes, you’ll have to maneuver yourself into position before you can gather information. For example, you might have to \kw{Prowl} to a good hiding place first and then \kw{Study} the cultists when they perform their dark ritual.

\subsection{Investigation}

Some questions are too complex to answer immediately with a single gather information roll. For instance, you might want to discover the network of contraband smuggling routes in the city. In these cases, the GM will tell you to start a \textbf{long-term project} that you work on during \textbf{downtime}.

You track the investigation project using a progress clock. Once the clock is filled, you have the evidence you need to ask several questions about the subject of your investigation as if you had great effect.

\subsection{Examples \& Questions}
\begin{itemize}
\item You might \kw{Attune} to see echoes of recent spirit activity. \emph{Have any new ghosts been here? How can I find the spirit well that’s calling to them? What should I be worried about?}
\item You might \kw{Command} a local barkeep to tell you what he knows about the secret meetings held in his back room. \emph{What’s really going on here? What’s he really feeling about this? Is he part of this secret group?}
\item You might \kw{Consort} with a well-connected friend to learn secrets about an enemy, rival, or potential ally. \emph{What do they intend to do? What might I suspect about their motives? How can I discover leverage to manipulate them?}
\item You might \kw{Hunt} a courier across the city, to discover who’s receiving satchels of coin from a master duelsit. \emph{Where does the package end up? How can I find out who signed for the package at City Hall?}
\item You might \kw{Study} ancient and obscure books to discover an arcane secret. \emph{How can I disable the runes of warding? Will anyone sense if they’re disabled?}
\item Or you might \kw{Study} a person to read their intentions and feelings. \emph{What are they really feeling? How could I get them to trust me?}
\item You might \kw{Survey} a manor house to case it for a heist. \emph{What’s a good point of infiltration? What’s the danger here?}
\item Or you might \kw{Survey} a charged situation when you meet another clan. \emph{What’s really going on here? Are they about to attack us?}
\item You might \kw{Sway} a powerful lord at a party so he divulges his future plans. \emph{What does he intend to do? How can I get him to think I might be a good partner in this venture?}
\item Or you might \kw{Sway} his bodyguard to confide in you about recent events. \emph{Where has he been lately? Who’s he been meeting with?}
\end{itemize}

\gbox{Gather Information}{
\emph{Ask a question and make an action roll or a fortune roll. The GM answers you honestly, with a level of detail depending on the effect level.}

\begin{description}
\item[Great] You get exceptional details. The information is complete and follow-up questions may expand into related areas or reveal more than you hoped for.
\item [Standard] You get good details. Clarifying and follow-up questions are possible.
\item [Limited] You get incomplete or partial information. More information gathering will be needed to get all the answers.
\end{description}
}

\section{Coin \& Stash}


\subsection{Coin}

\kw{Coin} is an abstract measure of cash and liquid assets.

The few bits PCs use in their daily lives are not tracked. If a character wants to spend to achieve a small goal (bribe a doorman), use the PC’s \textbf{lifestyle quality} for a fortune roll.

\subsection{Monetary values}

\begin{description}
\item[1 \kw{Coin}] A full purse of silver pieces. A week’s wages for a merchant.
\item[2 \kw{Coin}] A fine weapon. A weekly income for a small business. A fine piece of art. A set of luxury clothes.
\item[4 \kw{Coin}] A satchel full of silver. A month’s wages.
\item[6 \kw{Coin}] An exquisite jewel. A heavy burden of silver pieces.
\item[8 \kw{Coin}] A good monthly take for a small business. A small safe full of coins and valuables. A very rare luxury commodity.
\item[10 \kw{Coin}] Liquidating a significant asset—a carriage and goats, a horse, a deed to a small property.
\end{description}

More than 4 \kw{Coin} is an impractical amount to keep lying around. You must spend the excess or put it in your \textbf{stash} (see below). A clan can also store 4 \kw{Coin} in their lair, by default. If they upgrade to a \textbf{vault}, they can expand their stores to 8 and then 16 \kw{Coin}. Any \kw{Coin} beyond their limit must be spent as soon as possible (typically before the next score) or distributed among the clan members.

One unit of \kw{Coin} in silver pieces or other bulk currency takes up one item slot for your load when carried.

\subsection{Coin Use}
\begin{itemize}
\item Spend 1 \kw{Coin} to get an additional activity during downtime.
    Spend 1 \kw{Coin} to increase the result level of a downtime activity roll.
    Spend \kw{Coin} to avoid certain clan entanglements.
    Put \kw{Coin} in your character’s stash to improve their lifestyle and circumstances when they retire. See the next page.
    Spend \kw{Coin} when you advance your clan’s Tier.
\end{itemize}
\subsection{Stash \& Retirement}

When you mark your character’s final \kw{Trauma} and they retire, the amount of \kw{Coin} they’ve managed to stash away determines their fate. Your stash tracker is on your character sheet.
\begin{description}
\item[Stash 0-10: Poor soul] You end up in the gutter, awash in vice and misery.
\item[Stash 11-20: Meager] A tiny hovel that you can call your own.
\item[Stash 21-39: Modest] A simple home or apartment, with some small comforts. You might operate a tavern or small business.
\item[Stash 40: Fine] A well-appointed home or apartment, claiming a few luxuries. You might operate a medium business.
\end{description}

In addition, each full row of stash (10 \kw{Coins}) indicates the \textbf{quality level of the character’s lifestyle}, from zero (street life) to four (luxury).

\begin{qb}Cross wants some alone-time with a prospective new friend, but he can’t take them back to the hidden lair where he lives, so what to do? Ryan, Cross’s player, says he wants to rent a nice room for the evening, so the GM asks for a fortune roll using Cross’s lifestyle rating to see what quality of room Cross can manage.\end{qb}

\subsubsection{Removing \kw{Coin} from your stash}

If you want to pull \kw{Coin} out of your stash, you may do so, at a cost. Your character sells off some of their assets and investments in order to get some quick cash. \textbf{For every 2 stash removed, you get 1 \kw{Coin} in cash.}

\section{The Clan Game}

\subsection{Caste}

Tsoly\'anu has a rigid caste system based upon clan or family membership. Ones \kw{Caste} is fixed at birth and is not subject to change. Clans by \kw{Caste} are as follows:

\begin{description}
\item[Imperial] The Tlakot\'ani God-Kings. One cannot bribe or marry into the imperial clan one can only be born into the imperial clan.  Mostly well-to-do farmers and middle-class urban merchants surrounding B\'ey S\"u, Hauma, and Usen\'anu.
\item[Very High] This is the nobility class and then tend to hold high ranking administrative officies, high circle priests, landowners, bankers, and imperial officers and any profession felt to be noble. They will be arrogant even when poor. Example clans are the Blade Raised High, Cloak of Azure Gems, Golden Suburst, \'Ito, Jade Diadem, Might of G\'anga, Golden Bough, Sea Blue, Sword of Fire, and Vr\'iddi. Each with their own rich history.
\item[High] Similar in profile to the Very High. They also include scholars, scribes and physicians. Larger clans include Amber Cloak, Dark Fear, Dark Flame, Dark Moon, Dark Water, Domed Tomb, Emerald Girdle, Great Stone, Grey Cloak, Grey Want, High Pinnacle, Iron Helm, Joyous of Vr\'a, Purple Gem,  Red Mountain, Red Stone, Red Sword, Red Stone, Rising Sun, Scarlet Sail, Standing Stone, Sweet Singers of Nakom\'e, Staff of Beneficence, White Crystal, and White Stone
\item[Medium] Mid-level bureaucrats, mid-circle priests, sailors and professional crafts. Solid and honorable, treated respectfullly, but left out of politics. Clans include Black Hood, Black Monolith, Black Mountain, Black Pinnacle, Blazoned Sail, Blue Kirtle, Blue Shadow, Blue Stone, Blue Stream, Broken Bough, Broken Reed, Copper Door, First Moon, Glory of the Worm, Green Bough, Green Kirtle, Green Malachite, Golden Dawn, Golden Lintel, Golden Sapphire, Golden Sheaf, Golden Sphere, Iron Fist, Iron Hand, Moon of Evening, Red Eye of Dawn, Red Flower, Red Sky, Red Star, Ripened Sheaf, Scroll of Wisdom, Silver Collar, Silver Lightning, Standing Pinnacle, Victorious Globe, Weeping Stone
\item[Low]
\item[Very Low]
\item[Clanless] Nakom\'e. Those without a clan are viewed as the lowest barbarians and are treated as such. Well to do foreigners will get a modicum of respect but in general are considered lower than slaves. Marriage or obvious valuable talents can earn one acceptance into a clan as ``outside blood''. 
\end{description}

\subsection{Tier}

Each notable faction is ranked by \kw{Tier}—a measure of wealth, influence, and scale. At the highest level are the \kw{Tier} V and VI factions, the true powers of the city. Your crew begins at \kw{Tier} 0.

You’ll use your \kw{Tier} rating to roll dice when you acquire an asset, as well as for any fortune roll for which your crew’s overall power level and influence is the primary trait. Most importantly, your Tier determines the quality level of your items as well as the quality and scale of the gangs your crew employs—and thereby what size of enemy you can expect to handle.

\subsubsection{Clan scale by Tier}

A clan's size in a town or city is measured by it's \kw{Tier}. This size does not include any hired or slave labour. 

\begin{description}
\item[Tier V] Massive. (80+ people)
\item[Tier IV] Huge. (40 people)
\item[Tier III] Large. (20 people)
\item[Tier II] Medium. (12 people)
\item[Iter I] Small. (3-6 people)
\item[Tier 0] Tiny. (1 or 2 people)
\end{description}

\subsection{Hold}

On the clan ladder next to the \kw{Tier} number is a letter indicating the strength of each faction's \kw{Hold}. Hold represents how well a faction can maintain their current position on the ladder. W indicates \textbf{weak} hold. S indicates \textbf{strong} hold. Your crew begins wtih \textbf{strong} hold at \kw{Tier 0}.

\subsection{Development}

To move up the ladder and develop your crew, you need \kw{Rep}. \kw{Rep} is a measure of clout and renown. When you accrue enough \kw{Rep}, the other factions take you more seriously and you attract the support needed to develop and grow.

When you complete a score, your crew earns \kw{2 Rep}. If the target of the score is higher Tier than your crew, you get \textbf{+1 \kw{Rep} per \kw{Tier} higher}. If the target of the score is lower \kw{Tier}, you get \kw{-1 Rep} per \kw{Tier} lower (minimum zero).

You need \kw{12 Rep} to fill the \kw{Rep} tracker on your crew sheet. When you fill the tracker, do one of the following:

\begin{itemize}
\item If your hold is weak, it becomes strong. \textbf{Reset your rep to zero}.
\item If your hold is strong, you can pay to increase your crew Tier by one. This costs \kw{Coin} equal to your \textbf{new \kw{Tier} x 8}. As long as your \kw{Rep} tracker is full, you don’t earn new \kw{Rep} (12 is the max). Once you pay and increase your \kw{Tier}, \textbf{reset your \kw{Rep} to zero and reduce your hold to weak}.
\end{itemize}

\subsection{Turf}

Another way to contribute to the crew’s development is by acquiring \kw{Turf}. When you seize and hold territory, you establish a more stable basis for your \kw{Rep}. Each piece of turf that you claim represents abstracted support for the crew (often a result of the fear you instill in the citizens on that turf).

Turf is marked on your rep tracker (see the example below). Each piece of turf you hold reduces the rep cost to develop by one. So, if you have 2 turf, you need 10 rep to develop. If you have 4 \kw{Turf}, you need 8 \kw{Rep} to develop. You can hold a maximum of 6 \kw{Turf}. When you develop and reset your \kw{Rep}, you keep the marks from all the \kw{Turf} you hold.

\begin{center}\includegraphics[scale=0.2]{img/turf.png}\end{center}

\begin{qb}
If you hold 3 pieces of \kw{Turf}, you need only 9 \kw{Rep} to develop, instead of 12. 

When you develop, you’ll clear the 9 \kw{Rep} marks, but keep the 3 \kw{Turf} marks. Mark \kw{Turf} from the right side, to show the ``cap'' on how much \kw{Rep} is needed.
\end{qb}

Also, when one acquires \kw{Turf}, one expands the scope of the clan's \textbf{hunting grounds}.

\subsubsection{Reducing Hold}

You may perform an operation specifically to reduce the hold of another faction, if you know how they’re vulnerable. If the operation succeeds, the target faction loses 1 level of hold. If their hold is weak and it drops, the faction loses 1 \kw{Tier} and stays weak.

When a faction is at war, it temporarily loses 1 hold.

Your crew can lose hold, too, following the same rules above. If your crew is \kw{Tier} 0, with weak hold, and you lose hold for any reason, your lair comes under threat by your enemies or by a faction seeking to profit from your misfortune.

\subsection{Faction Status}

One's crew’s \kw{Status} with each faction indicates how well one is liked or hated. This is different than the \kw{Caste} which commands respect. A clan may be of noble \kw{Caste} and command respect in society but be utterly destested. Status is rated from -3 to +3, with zero (neutral) being the default starting status. Track one's status with each faction on the faction sheet.

When you create ones crew, assign some positive and negative status ratings to reflect recent history. The ratings will then change over time based on actions in play.

\subsubsection{Faction Status Changes}

When one executes an operation, one gains -1 or -2 \kw{Status} with factions that are hurt by ones actions. One may also gain +1 \kw{Status} with a faction that your operation helps. (If one keeps the operation completely quiet then \kw{Status} doesn’t change.)  One's \kw{Status} may also change if you do a favor for a faction or if you refuse one of their demands.

\subsubsection{Faction Status Levels}

\begin{description}
\item[+3: Allies] This faction will help you even if it’s not in their best interest to do so. They expect you to do the same for them.
\item[+2: Friendly] This faction will help you if it doesn’t create serious problems for them. They expect you to do the same.
\item[+1: Helpful] This faction will help you if it causes no problems or significant cost for them. They expect the same from you.
\item[0: Neutral]
\item[-1: Interfering] This faction will look for opportunities to cause trouble for you (or profit from your misfortune) as long as it causes no problems or significant cost for them. They expect the same from you.
\item[-2: Hostile] This faction will look for opportunities to hurt you as long as it doesn’t create serious problems for them. They expect you to do the same, and take precautions against you.
\item[-3: War] This faction will go out of its way to hurt you even if it’s not in their best interest to do so. They expect you to do the same, and take precautions against you. When you’re at war with any number of factions, your crew suffers +1 heat from scores, temporarily loses 1 hold, and PCs get only one downtime action rather than two. You can end a war by eliminating your enemy or by negotiating a mutual agreement to establish a new \kw{Status} rating.
\end{description}

\begin{qb}If your crew has weak hold when you go to war, the temporary loss of hold causes you to lose one Tier. When the war is over, restore your crew’s Tier back to its pre-war level.\end{qb}

\subsection{Claims}

Each clan sheet has a map of claims available to be seized. The claim map displays a default roadmap for your clan type. Claims should usually be seized in an orderly sequence, by following the paths from the central square, the clanhouse.

\textit{However, you may attempt to seize any claim on your map}, ignoring the paths (or even seek out a special claim not on your map) but these operations will always be especially difficult and require exceptional efforts to discover and achieve.

\subsubsection{Seizing a claim}

Every claim is already controlled by a faction. To acquire one for yourself, you have to take it from someone else. To seize a claim, tell the GM which claim on your map your crew intends to capture. The GM will detail the claim with a location and a description and will tell you which faction currently controls that claim. Or the GM might offer you a choice of a few options if they’re available.

If you choose to ignore the roadmap paths when seizing a claim, the GM might tell you that you’ll need to investigate and gather information in order to discover a claim of that type before you can attempt to seize it.

Execute the operation like any other \textbf{score}, and if you succeed, you seize the claim and the targeted faction loses the claim.

Seizing a claim is a serious attack on a faction, usually resulting in -2 faction status with the target, and potentially +1 status with its enemies.

As soon as you seize a claim, you enjoy the listed benefit for as long as you hold the claim. Some claims count as \kw{Turf}. Others provide special benefits to the clan, such as bonus dice in certain circumstances, extra \kw{Coin} generated for the clan’s treasury, or new opportunities for action.

\subsubsection{Losing a claim}

An enemy faction may try to seize a claim that your crew holds. You can fight to defend it, or negotiate a deal with the faction, depending on the situation. If you lose a claim, you lose all the benefits of that claim. If your lair is lost, you lose the benefits of all of your claims until you can restore your lair or establish a new one. To restore or establish a new lair, accomplish a score to do so.

\section{Advancement}


\subsection{PC Advancement}

Each player keeps track of the experience points (\kw{xp}) that their character earns.

During the game session, mark \kw{xp}:

    When one makes a \textbf{desperate action roll}. Mark 1 \kw{xp} in the attribute for the action rolled. For example, if rolling a desperate \kw{Skirmish} action, one marks \kw{xp} in \kw{Prowess}. When one rolls in a \textbf{group action} that’s desperate, also mark \kw{xp}.

At the end of the session, review the \kw{xp} \textbf{triggers} on your character sheet. For each one, mark 1 \kw{xp} if it happened at all, or mark 2 \kw{xp} if it happened a lot during the session. The \kw{xp} triggers are:

\begin{description}
\item[Playbook-specific \kw{xp} trigger] For example ``\textit{Address a challenge with violence or coercion.}'' To ``address a challenge,'' one's character should attempt to overcome a tough obstacle or threat. It doesn’t matter if the action is successful or not. You get \kw{xp} either way.
\item[Expressed your beliefs, drives, heritage, or background] One's character’s beliefs and drives are ones to define, session to session. Feel free to tell the group about them when one marks \kw{xp}.
\item[Struggled with issues from your vice or traumas] Mark \kw{xp} for this if your vice tempted you to some bad action or if a trauma condition caused you trouble. Simply indulging your vice doesn’t count as struggling with it (unless you overindulge).
\end{description}

One may mark end-of-session \kw{xp} on any \kw{xp} track (any attribute or one's playbook \kw{xp} track).

When \kw{xp} track is filled, clear all the marks and take an advance. When one takes an advance from the playbook track, you may choose an additional special ability. When taking an advance from an attribute, one may add an additional action dot to one of the actions under that attribute.

\begin{qb}Nadja is playing a Hound. At the end of the session, she reviews her \kw{xp} triggers and tells the group how much \kw{xp} she’s getting. She rolled two desperate \kw{Hunt} actions during the session, so she marked 2 \kw{xp} on her \kw{Insight} \kw{xp} track. She addressed several challenges with tracking or violence, so she marks 2 \kw{xp} for that. She expressed her heritage many times when dealing with the gang from her homeland, so she takes 2 \kw{xp} for that. She also showcased her character’s beliefs, but 2 \kw{xp} is the maximum for that category, so she doesn’t get any more. She didn’t struggle with her vice or traumas, so no \kw{xp} there. That’s 4 \kw{xp} at the end of the session. She decides to put it all in her \kw{Insight} \kw{xp} track. This fills the track, so she adds a new action dot in \kw{Hunt}.\end{qb}

You can also earn \kw{xp} by \textbf{training} during downtime. When you train, mark \kw{xp} in one of your attributes or in your playbook. A given \kw{xp} track can be trained only once per downtime phase.

\subsection{Clan Advancement}

At the end of the session, review the crew \kw{xp} triggers and mark 1 crew \kw{xp} for each item that occurred during the session. If an item occurred multiple times or in a major way, mark 2 crew \kw{xp} for it. The crew \kw{xp} triggers are:

\begin{description}
\item[Crew-specific \kw{xp} trigger] For example, the Smugglers’ is ``Execute a smuggling operation or acquire new clients or contraband sources.'' If the crew successfully completed an operation from this trigger, mark xp.
\item[Contend with challenges above your current station] If you tangled with higher \kw{Tiers} or more dangerous opposition, mark \kw{xp} for this.
\item[Bolster your crew’s reputation or develop a new one] Review one's crew’s reputation. Did one do anything to promote it? Also mark \kw{xp} if one has developed a new reputation for the crew.
\item[Express the goals, drives, inner conflict, or essential nature of the crew] This one is very broad! Essentially, did anything happen that highlighted the specific elements that make one's crew unique?
\end{description}

\textbf{When you fill your crew advancement tracker}, clear the marks and take a new \textbf{special ability} or mark \textbf{two crew upgrade boxes}.

\begin{qb}For example, when a clan of Assassins earns a clan advance, they could take a new special ability, like \kw{Predators}. Or they could mark two upgrades, like \kw{Rigging} and \kw{Resolve Training}.\end{qb}

Say how one has obtained this new ability or upgrades for the clan. \textit{Where did it come from? How does it become a new part of the clan?}

\subsubsection{Profits}

Every time the crew advances, \textbf{each PC gets stash} equal to the clan \kw{Tier}+2, to represent profits generated by the crew as they’ve been operating.


