\chapter{Basics}

\section{Guild Initiation}

\textit{A busy stage full of books shelves is surrounded by cameramen and boom mike operators all surrounding a well dressed
gentlemen in an overstuffed chair.}

\textit{Ready. Set. Action.}

\textsc{Player:} Good Evening, I am Mr. Player, esq. and you have chosen an excellent time to join use for the next installment of `The Guild'. Tonights episode features a conflict between opposing forces, sometimes there will be trajedy and sometimes there will be comedy driven by the divine satyr of drama.

\textsc{Director:} Cut, cut. What is the gaffer doing in the shot?

\textsc{Gaffer:} Uh, I just got a bomb threat for the building and I noticed there appears to be a claymore mine hanging under Mr. Player's chair. I wouldn't stand up if I were you Mr. Player, I think it's rigged to go off when you stand.

\textit{Panic ensues, as most of the crew drops their equipment and flees the building leaving just Mr. Player and a nervous looking Gaffer who happens to be in the spotlight.}

\textsc{Director:} Okay, the conflict is established. Mr. Player's life is on the line and you've chosen to resolve--but what is your action.

\textsc{Gaffer:} I `fought in the Iraq war' so I'm going to try and disable the mine. (\textit{Smash Cut: The Gaffer is sweating in the desert disarming a roadside improvised explosive device. Cut back and the same sweat is on his brow.}) 

\textsc{Director:} Move the chip to center off your advantage card, that grants you a +2 on the outcome. It's a passive opposition check and the difficulty is Fair (the antagonist appears to have rushed the job) so that's a -1 for a +1 total. Also, the chip from the ``Relationship: You owe me big time.'' can be moved to the pool as well. Roll the Fudge dice and let's look at the outcome.

\textsc{Gaffer:} Ouch! I got a -3, with the +1 bonus that's a minor failure with minor consequence.

\textsc{Director:} Take a black trajegy chip from the pool, and a consequence card and tell us what happens.

\textsc{Gaffer:}\textit{Takes the chip.} I manage to stop the trigger pin from engaging, but only as long as I'm holding it. Mr. Player abandons me and runs when I tell him go. I then let go and make a run for it. The explosion goes off and the shrapnel hits me in the leg.

\textit{The Gaffer takes a card and writes `Minor Consequence: Wounded Leg' on it.}

\vspace{5mm}

\lettrine{\initbg T\llap{\initfg T}}{he} Guild is a role playing game, a social game where a bunch of friends get together and create an
interactive story. One that focuses on narrative development with some dependency on random outcomes but avoiding excessive tables. The progenator of modern role playing is Dungeons \& Dragons by Gary Gygax. Players would assume Tolkein-esque characters and fight things such as orks and goblins and it grew out of miniature wargaming. As such it was heavily dice and table based. There is a tension between the love of dice table driven combat and those who claim to seek \emph{true} role playing and not \emph{roll} playing.
  
Both have their merits. Many role players have treasured moments 
of surprise when the dice resulted in a near impossible outcome in which the whole table erupted
in shouts and cheers; That feeling of success against insurmountable odds filling the air. There
have been other experiences when the players constructed a narrative over time that became so
intriguing to those involved that there collective story is remembered fondly decades later. The Guild promises both, and a means to continue a campaign over time.

A Guild game also allows for what was once a plot driven monorail to become a garden of forking paths. Many a \kw{gamemaster} of a role playing game has been frustrated by carefully planning an adventure with a fairly linear structure only to watch it go off the rails immediately and struggle to adapt. Why not 
put some of that effort back on the players and make it collaborative. The Guild puts players in some control of the narrative while providing very wide and unknown (to the players) boundaries of the plot. How the game develops and unfolds wil be vastly different for every group, with each player having their own unique story of successes and failures. The Guild also strives to provide really simple choices to prevent player confusion with means to stoke and unlock creativity among those who don't feel so creative.

\section{Materials Required}

The needs to start a game are relatively simple as follows:

\begin{itemize}
\item Three to six players with two to four hours to share. One player agrees to be the \kw{gamemaster} and the others the \kw{troupe} of players.
\item Four times the number of players plus a few extra poker chips half in one color and half in another, preferably black and white. These will be used to track player position towards the final outcome of the conflict, and which \kw{characteristics}
\marginpar{Characteristics define a character. They can be \kw{relationships} or \kw{attributes}.}
have been invoked during a session.
\item Index cards and a suitable writing implement. These will be used to construct player decks of \textbf{characteristics} and are the core bookkeeping component of a session.
\item Fate dice, at least four, are used to consult the bionomial goddess and see what fate she bestows upon any give plot thread. Fate dice are six-sided dice that have two side blank, two sides are a plus and two side that are a minus. With four dice the potential outcome is from negative four to four. Regular six sided dice can be used, just treat the one and two as a minus and the five and six as a plus. 

\end{itemize}

\section{Outline of Play}

\subsection{In a World}

The gamemaster reads the description of the world and setting and describes the main conflict. It could be a pre-existing setting, or a joint visioning between the players. \marginpar{Microscope by Ben Robbins is a recommended world building
game that could establish the conflict to play with The Guild.}

\subsection{Casting of Characters}

\subsubsection{Setup}

The gamemaster distributes any player character decks from previous games in a campaign to players. Each player including the \kw{gamemaster} needs two \kw{relationship} cards between them, a \kw{psyche} and a \kw{physical} card. If a players decks contains any of these relating to either player to the left or right, these are placed first. Any conflicts in choices are resolved in turn order.\marginpar{Turn order starts with the player to the left of the \kw{gamemaster} and proceeds clockwise} Cards otherwise blank labeled \kw{psyche} and a \kw{physical} are now placed to fill the gaps. If a player has existing \kw{attributes} and \kw{title} card he can place these in front of him at this time. In adddition, the \kw{gamemaster} may place \kw{secret} cards in front of his area on the table.

\subsubsection{Relationships}

The \kw{gamemaster} now provides a chart that allows players to select \kw{categories} for each of the blank relationship cards. In turn order players select a \kw{category} and add it to any blank card.

Once all these are filled, the \kw{gamemaster} provides four tables that provide \kw{details} for \kw{relationships}. Continuing with the next player in turn order, these are filled in.

\subsubsection{Character}

Next each player either displays or creates a new \kw{title} card. This has the name of the character, a title or blurb, and any details to round out the character. In turn order each player gives a short introduction to the group of their charcter.

\subsubsection{Attributes}

Each player needs a \kw{strength} and \kw{weakness} \kw{attribute} if they do not already have one from a prior game. Any player missing a \kw{strength} is given one by the player to their right. Any player missing a \kw{weakness} is given one by the player to their left. 

\subsubsection{Tracking Tokens} Each player is given two white and two black tokens (or suitable differing colors) that they place upon their attributes and relationships anyway they desire.

\subsection{Exposition} The gamemaster reads the opening flavor text that describes the opening conflict or goal that the players needs to address.

\subsection{Rising Actions} Each player takes a turn, starting with the player to the left of the gamemaster. On their turn they can establish or resolve how the narrative revolves around their player. The rest of the players will take the opposite duty, with the gamemaster responsible for keeping the game flowing and vetoing anything that conflicts with secret information. The GM must reveal the contents of a secret card that would be violated by the proposal to the player and that player can modify his actions appropriatly. Any relationships between players mentioned get their tokens moved to the central pool. A dice throw is made to determine outcome, and any attributes used to modifying the outcome are moved to the center pool. The outcome table is consulted, and the resolving group describes the outcome. Consequence cards may be assigned. If the outcome was positive, the player takes a white chip, if the outcome was negative the player gets a black chip.

\textsc{Note:} It is desireable at the end to have only a single color of chip. 

Play continues until all players have had two turns. The gamemaster gets the last turn and has to weave a story that brings all the action towards a climax that drives the central conflict forward.

\subsection{Tilt}

The player with the most white tokens choses a tilt from the available for the scenario and writes it on a note card and adds that to the middle of the table and places a spare white token on it. Likewise the player with the most black tokens does the same.

\subsection{Falling Action}

Play proceeds exactly as the rising action, and the tilt aspects are available for invocation as well. 

\subsection{Resolution}

Each player removes all black white chip pairs from their pool. The remaining determines their modifier on the outcomes table. Each player in turn rolls their outcome, and applies resulting consequences or benefits bestowed. Next in turn order each person returns a single remaining chit to the pool and narrates a fact about his final fate in the conflict. If they have no remaining chips then they are skipped.

\subsection{Cleanup}

Next each player takes their player card and puts under it any advantages, disadvantages, benefits, consequences, and then the relationships to their left. All of these are then collected in a stack for use in the next game of the campaign.